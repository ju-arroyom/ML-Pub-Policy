
% Default to the notebook output style

    


% Inherit from the specified cell style.




    
\documentclass[11pt]{article}

    
    
    \usepackage[T1]{fontenc}
    % Nicer default font (+ math font) than Computer Modern for most use cases
    \usepackage{mathpazo}

    % Basic figure setup, for now with no caption control since it's done
    % automatically by Pandoc (which extracts ![](path) syntax from Markdown).
    \usepackage{graphicx}
    % We will generate all images so they have a width \maxwidth. This means
    % that they will get their normal width if they fit onto the page, but
    % are scaled down if they would overflow the margins.
    \makeatletter
    \def\maxwidth{\ifdim\Gin@nat@width>\linewidth\linewidth
    \else\Gin@nat@width\fi}
    \makeatother
    \let\Oldincludegraphics\includegraphics
    % Set max figure width to be 80% of text width, for now hardcoded.
    \renewcommand{\includegraphics}[1]{\Oldincludegraphics[width=.8\maxwidth]{#1}}
    % Ensure that by default, figures have no caption (until we provide a
    % proper Figure object with a Caption API and a way to capture that
    % in the conversion process - todo).
    \usepackage{caption}
    \DeclareCaptionLabelFormat{nolabel}{}
    \captionsetup{labelformat=nolabel}

    \usepackage{adjustbox} % Used to constrain images to a maximum size 
    \usepackage{xcolor} % Allow colors to be defined
    \usepackage{enumerate} % Needed for markdown enumerations to work
    \usepackage{geometry} % Used to adjust the document margins
    \usepackage{amsmath} % Equations
    \usepackage{amssymb} % Equations
    \usepackage{textcomp} % defines textquotesingle
    % Hack from http://tex.stackexchange.com/a/47451/13684:
    \AtBeginDocument{%
        \def\PYZsq{\textquotesingle}% Upright quotes in Pygmentized code
    }
    \usepackage{upquote} % Upright quotes for verbatim code
    \usepackage{eurosym} % defines \euro
    \usepackage[mathletters]{ucs} % Extended unicode (utf-8) support
    \usepackage[utf8x]{inputenc} % Allow utf-8 characters in the tex document
    \usepackage{fancyvrb} % verbatim replacement that allows latex
    \usepackage{grffile} % extends the file name processing of package graphics 
                         % to support a larger range 
    % The hyperref package gives us a pdf with properly built
    % internal navigation ('pdf bookmarks' for the table of contents,
    % internal cross-reference links, web links for URLs, etc.)
    \usepackage{hyperref}
    \usepackage{longtable} % longtable support required by pandoc >1.10
    \usepackage{booktabs}  % table support for pandoc > 1.12.2
    \usepackage[inline]{enumitem} % IRkernel/repr support (it uses the enumerate* environment)
    \usepackage[normalem]{ulem} % ulem is needed to support strikethroughs (\sout)
                                % normalem makes italics be italics, not underlines
    

    
    
    % Colors for the hyperref package
    \definecolor{urlcolor}{rgb}{0,.145,.698}
    \definecolor{linkcolor}{rgb}{.71,0.21,0.01}
    \definecolor{citecolor}{rgb}{.12,.54,.11}

    % ANSI colors
    \definecolor{ansi-black}{HTML}{3E424D}
    \definecolor{ansi-black-intense}{HTML}{282C36}
    \definecolor{ansi-red}{HTML}{E75C58}
    \definecolor{ansi-red-intense}{HTML}{B22B31}
    \definecolor{ansi-green}{HTML}{00A250}
    \definecolor{ansi-green-intense}{HTML}{007427}
    \definecolor{ansi-yellow}{HTML}{DDB62B}
    \definecolor{ansi-yellow-intense}{HTML}{B27D12}
    \definecolor{ansi-blue}{HTML}{208FFB}
    \definecolor{ansi-blue-intense}{HTML}{0065CA}
    \definecolor{ansi-magenta}{HTML}{D160C4}
    \definecolor{ansi-magenta-intense}{HTML}{A03196}
    \definecolor{ansi-cyan}{HTML}{60C6C8}
    \definecolor{ansi-cyan-intense}{HTML}{258F8F}
    \definecolor{ansi-white}{HTML}{C5C1B4}
    \definecolor{ansi-white-intense}{HTML}{A1A6B2}

    % commands and environments needed by pandoc snippets
    % extracted from the output of `pandoc -s`
    \providecommand{\tightlist}{%
      \setlength{\itemsep}{0pt}\setlength{\parskip}{0pt}}
    \DefineVerbatimEnvironment{Highlighting}{Verbatim}{commandchars=\\\{\}}
    % Add ',fontsize=\small' for more characters per line
    \newenvironment{Shaded}{}{}
    \newcommand{\KeywordTok}[1]{\textcolor[rgb]{0.00,0.44,0.13}{\textbf{{#1}}}}
    \newcommand{\DataTypeTok}[1]{\textcolor[rgb]{0.56,0.13,0.00}{{#1}}}
    \newcommand{\DecValTok}[1]{\textcolor[rgb]{0.25,0.63,0.44}{{#1}}}
    \newcommand{\BaseNTok}[1]{\textcolor[rgb]{0.25,0.63,0.44}{{#1}}}
    \newcommand{\FloatTok}[1]{\textcolor[rgb]{0.25,0.63,0.44}{{#1}}}
    \newcommand{\CharTok}[1]{\textcolor[rgb]{0.25,0.44,0.63}{{#1}}}
    \newcommand{\StringTok}[1]{\textcolor[rgb]{0.25,0.44,0.63}{{#1}}}
    \newcommand{\CommentTok}[1]{\textcolor[rgb]{0.38,0.63,0.69}{\textit{{#1}}}}
    \newcommand{\OtherTok}[1]{\textcolor[rgb]{0.00,0.44,0.13}{{#1}}}
    \newcommand{\AlertTok}[1]{\textcolor[rgb]{1.00,0.00,0.00}{\textbf{{#1}}}}
    \newcommand{\FunctionTok}[1]{\textcolor[rgb]{0.02,0.16,0.49}{{#1}}}
    \newcommand{\RegionMarkerTok}[1]{{#1}}
    \newcommand{\ErrorTok}[1]{\textcolor[rgb]{1.00,0.00,0.00}{\textbf{{#1}}}}
    \newcommand{\NormalTok}[1]{{#1}}
    
    % Additional commands for more recent versions of Pandoc
    \newcommand{\ConstantTok}[1]{\textcolor[rgb]{0.53,0.00,0.00}{{#1}}}
    \newcommand{\SpecialCharTok}[1]{\textcolor[rgb]{0.25,0.44,0.63}{{#1}}}
    \newcommand{\VerbatimStringTok}[1]{\textcolor[rgb]{0.25,0.44,0.63}{{#1}}}
    \newcommand{\SpecialStringTok}[1]{\textcolor[rgb]{0.73,0.40,0.53}{{#1}}}
    \newcommand{\ImportTok}[1]{{#1}}
    \newcommand{\DocumentationTok}[1]{\textcolor[rgb]{0.73,0.13,0.13}{\textit{{#1}}}}
    \newcommand{\AnnotationTok}[1]{\textcolor[rgb]{0.38,0.63,0.69}{\textbf{\textit{{#1}}}}}
    \newcommand{\CommentVarTok}[1]{\textcolor[rgb]{0.38,0.63,0.69}{\textbf{\textit{{#1}}}}}
    \newcommand{\VariableTok}[1]{\textcolor[rgb]{0.10,0.09,0.49}{{#1}}}
    \newcommand{\ControlFlowTok}[1]{\textcolor[rgb]{0.00,0.44,0.13}{\textbf{{#1}}}}
    \newcommand{\OperatorTok}[1]{\textcolor[rgb]{0.40,0.40,0.40}{{#1}}}
    \newcommand{\BuiltInTok}[1]{{#1}}
    \newcommand{\ExtensionTok}[1]{{#1}}
    \newcommand{\PreprocessorTok}[1]{\textcolor[rgb]{0.74,0.48,0.00}{{#1}}}
    \newcommand{\AttributeTok}[1]{\textcolor[rgb]{0.49,0.56,0.16}{{#1}}}
    \newcommand{\InformationTok}[1]{\textcolor[rgb]{0.38,0.63,0.69}{\textbf{\textit{{#1}}}}}
    \newcommand{\WarningTok}[1]{\textcolor[rgb]{0.38,0.63,0.69}{\textbf{\textit{{#1}}}}}
    
    
    % Define a nice break command that doesn't care if a line doesn't already
    % exist.
    \def\br{\hspace*{\fill} \\* }
    % Math Jax compatability definitions
    \def\gt{>}
    \def\lt{<}
    % Document parameters
    \title{ML-PS3\_new}
    
    
    

    % Pygments definitions
    
\makeatletter
\def\PY@reset{\let\PY@it=\relax \let\PY@bf=\relax%
    \let\PY@ul=\relax \let\PY@tc=\relax%
    \let\PY@bc=\relax \let\PY@ff=\relax}
\def\PY@tok#1{\csname PY@tok@#1\endcsname}
\def\PY@toks#1+{\ifx\relax#1\empty\else%
    \PY@tok{#1}\expandafter\PY@toks\fi}
\def\PY@do#1{\PY@bc{\PY@tc{\PY@ul{%
    \PY@it{\PY@bf{\PY@ff{#1}}}}}}}
\def\PY#1#2{\PY@reset\PY@toks#1+\relax+\PY@do{#2}}

\expandafter\def\csname PY@tok@w\endcsname{\def\PY@tc##1{\textcolor[rgb]{0.73,0.73,0.73}{##1}}}
\expandafter\def\csname PY@tok@c\endcsname{\let\PY@it=\textit\def\PY@tc##1{\textcolor[rgb]{0.25,0.50,0.50}{##1}}}
\expandafter\def\csname PY@tok@cp\endcsname{\def\PY@tc##1{\textcolor[rgb]{0.74,0.48,0.00}{##1}}}
\expandafter\def\csname PY@tok@k\endcsname{\let\PY@bf=\textbf\def\PY@tc##1{\textcolor[rgb]{0.00,0.50,0.00}{##1}}}
\expandafter\def\csname PY@tok@kp\endcsname{\def\PY@tc##1{\textcolor[rgb]{0.00,0.50,0.00}{##1}}}
\expandafter\def\csname PY@tok@kt\endcsname{\def\PY@tc##1{\textcolor[rgb]{0.69,0.00,0.25}{##1}}}
\expandafter\def\csname PY@tok@o\endcsname{\def\PY@tc##1{\textcolor[rgb]{0.40,0.40,0.40}{##1}}}
\expandafter\def\csname PY@tok@ow\endcsname{\let\PY@bf=\textbf\def\PY@tc##1{\textcolor[rgb]{0.67,0.13,1.00}{##1}}}
\expandafter\def\csname PY@tok@nb\endcsname{\def\PY@tc##1{\textcolor[rgb]{0.00,0.50,0.00}{##1}}}
\expandafter\def\csname PY@tok@nf\endcsname{\def\PY@tc##1{\textcolor[rgb]{0.00,0.00,1.00}{##1}}}
\expandafter\def\csname PY@tok@nc\endcsname{\let\PY@bf=\textbf\def\PY@tc##1{\textcolor[rgb]{0.00,0.00,1.00}{##1}}}
\expandafter\def\csname PY@tok@nn\endcsname{\let\PY@bf=\textbf\def\PY@tc##1{\textcolor[rgb]{0.00,0.00,1.00}{##1}}}
\expandafter\def\csname PY@tok@ne\endcsname{\let\PY@bf=\textbf\def\PY@tc##1{\textcolor[rgb]{0.82,0.25,0.23}{##1}}}
\expandafter\def\csname PY@tok@nv\endcsname{\def\PY@tc##1{\textcolor[rgb]{0.10,0.09,0.49}{##1}}}
\expandafter\def\csname PY@tok@no\endcsname{\def\PY@tc##1{\textcolor[rgb]{0.53,0.00,0.00}{##1}}}
\expandafter\def\csname PY@tok@nl\endcsname{\def\PY@tc##1{\textcolor[rgb]{0.63,0.63,0.00}{##1}}}
\expandafter\def\csname PY@tok@ni\endcsname{\let\PY@bf=\textbf\def\PY@tc##1{\textcolor[rgb]{0.60,0.60,0.60}{##1}}}
\expandafter\def\csname PY@tok@na\endcsname{\def\PY@tc##1{\textcolor[rgb]{0.49,0.56,0.16}{##1}}}
\expandafter\def\csname PY@tok@nt\endcsname{\let\PY@bf=\textbf\def\PY@tc##1{\textcolor[rgb]{0.00,0.50,0.00}{##1}}}
\expandafter\def\csname PY@tok@nd\endcsname{\def\PY@tc##1{\textcolor[rgb]{0.67,0.13,1.00}{##1}}}
\expandafter\def\csname PY@tok@s\endcsname{\def\PY@tc##1{\textcolor[rgb]{0.73,0.13,0.13}{##1}}}
\expandafter\def\csname PY@tok@sd\endcsname{\let\PY@it=\textit\def\PY@tc##1{\textcolor[rgb]{0.73,0.13,0.13}{##1}}}
\expandafter\def\csname PY@tok@si\endcsname{\let\PY@bf=\textbf\def\PY@tc##1{\textcolor[rgb]{0.73,0.40,0.53}{##1}}}
\expandafter\def\csname PY@tok@se\endcsname{\let\PY@bf=\textbf\def\PY@tc##1{\textcolor[rgb]{0.73,0.40,0.13}{##1}}}
\expandafter\def\csname PY@tok@sr\endcsname{\def\PY@tc##1{\textcolor[rgb]{0.73,0.40,0.53}{##1}}}
\expandafter\def\csname PY@tok@ss\endcsname{\def\PY@tc##1{\textcolor[rgb]{0.10,0.09,0.49}{##1}}}
\expandafter\def\csname PY@tok@sx\endcsname{\def\PY@tc##1{\textcolor[rgb]{0.00,0.50,0.00}{##1}}}
\expandafter\def\csname PY@tok@m\endcsname{\def\PY@tc##1{\textcolor[rgb]{0.40,0.40,0.40}{##1}}}
\expandafter\def\csname PY@tok@gh\endcsname{\let\PY@bf=\textbf\def\PY@tc##1{\textcolor[rgb]{0.00,0.00,0.50}{##1}}}
\expandafter\def\csname PY@tok@gu\endcsname{\let\PY@bf=\textbf\def\PY@tc##1{\textcolor[rgb]{0.50,0.00,0.50}{##1}}}
\expandafter\def\csname PY@tok@gd\endcsname{\def\PY@tc##1{\textcolor[rgb]{0.63,0.00,0.00}{##1}}}
\expandafter\def\csname PY@tok@gi\endcsname{\def\PY@tc##1{\textcolor[rgb]{0.00,0.63,0.00}{##1}}}
\expandafter\def\csname PY@tok@gr\endcsname{\def\PY@tc##1{\textcolor[rgb]{1.00,0.00,0.00}{##1}}}
\expandafter\def\csname PY@tok@ge\endcsname{\let\PY@it=\textit}
\expandafter\def\csname PY@tok@gs\endcsname{\let\PY@bf=\textbf}
\expandafter\def\csname PY@tok@gp\endcsname{\let\PY@bf=\textbf\def\PY@tc##1{\textcolor[rgb]{0.00,0.00,0.50}{##1}}}
\expandafter\def\csname PY@tok@go\endcsname{\def\PY@tc##1{\textcolor[rgb]{0.53,0.53,0.53}{##1}}}
\expandafter\def\csname PY@tok@gt\endcsname{\def\PY@tc##1{\textcolor[rgb]{0.00,0.27,0.87}{##1}}}
\expandafter\def\csname PY@tok@err\endcsname{\def\PY@bc##1{\setlength{\fboxsep}{0pt}\fcolorbox[rgb]{1.00,0.00,0.00}{1,1,1}{\strut ##1}}}
\expandafter\def\csname PY@tok@kc\endcsname{\let\PY@bf=\textbf\def\PY@tc##1{\textcolor[rgb]{0.00,0.50,0.00}{##1}}}
\expandafter\def\csname PY@tok@kd\endcsname{\let\PY@bf=\textbf\def\PY@tc##1{\textcolor[rgb]{0.00,0.50,0.00}{##1}}}
\expandafter\def\csname PY@tok@kn\endcsname{\let\PY@bf=\textbf\def\PY@tc##1{\textcolor[rgb]{0.00,0.50,0.00}{##1}}}
\expandafter\def\csname PY@tok@kr\endcsname{\let\PY@bf=\textbf\def\PY@tc##1{\textcolor[rgb]{0.00,0.50,0.00}{##1}}}
\expandafter\def\csname PY@tok@bp\endcsname{\def\PY@tc##1{\textcolor[rgb]{0.00,0.50,0.00}{##1}}}
\expandafter\def\csname PY@tok@fm\endcsname{\def\PY@tc##1{\textcolor[rgb]{0.00,0.00,1.00}{##1}}}
\expandafter\def\csname PY@tok@vc\endcsname{\def\PY@tc##1{\textcolor[rgb]{0.10,0.09,0.49}{##1}}}
\expandafter\def\csname PY@tok@vg\endcsname{\def\PY@tc##1{\textcolor[rgb]{0.10,0.09,0.49}{##1}}}
\expandafter\def\csname PY@tok@vi\endcsname{\def\PY@tc##1{\textcolor[rgb]{0.10,0.09,0.49}{##1}}}
\expandafter\def\csname PY@tok@vm\endcsname{\def\PY@tc##1{\textcolor[rgb]{0.10,0.09,0.49}{##1}}}
\expandafter\def\csname PY@tok@sa\endcsname{\def\PY@tc##1{\textcolor[rgb]{0.73,0.13,0.13}{##1}}}
\expandafter\def\csname PY@tok@sb\endcsname{\def\PY@tc##1{\textcolor[rgb]{0.73,0.13,0.13}{##1}}}
\expandafter\def\csname PY@tok@sc\endcsname{\def\PY@tc##1{\textcolor[rgb]{0.73,0.13,0.13}{##1}}}
\expandafter\def\csname PY@tok@dl\endcsname{\def\PY@tc##1{\textcolor[rgb]{0.73,0.13,0.13}{##1}}}
\expandafter\def\csname PY@tok@s2\endcsname{\def\PY@tc##1{\textcolor[rgb]{0.73,0.13,0.13}{##1}}}
\expandafter\def\csname PY@tok@sh\endcsname{\def\PY@tc##1{\textcolor[rgb]{0.73,0.13,0.13}{##1}}}
\expandafter\def\csname PY@tok@s1\endcsname{\def\PY@tc##1{\textcolor[rgb]{0.73,0.13,0.13}{##1}}}
\expandafter\def\csname PY@tok@mb\endcsname{\def\PY@tc##1{\textcolor[rgb]{0.40,0.40,0.40}{##1}}}
\expandafter\def\csname PY@tok@mf\endcsname{\def\PY@tc##1{\textcolor[rgb]{0.40,0.40,0.40}{##1}}}
\expandafter\def\csname PY@tok@mh\endcsname{\def\PY@tc##1{\textcolor[rgb]{0.40,0.40,0.40}{##1}}}
\expandafter\def\csname PY@tok@mi\endcsname{\def\PY@tc##1{\textcolor[rgb]{0.40,0.40,0.40}{##1}}}
\expandafter\def\csname PY@tok@il\endcsname{\def\PY@tc##1{\textcolor[rgb]{0.40,0.40,0.40}{##1}}}
\expandafter\def\csname PY@tok@mo\endcsname{\def\PY@tc##1{\textcolor[rgb]{0.40,0.40,0.40}{##1}}}
\expandafter\def\csname PY@tok@ch\endcsname{\let\PY@it=\textit\def\PY@tc##1{\textcolor[rgb]{0.25,0.50,0.50}{##1}}}
\expandafter\def\csname PY@tok@cm\endcsname{\let\PY@it=\textit\def\PY@tc##1{\textcolor[rgb]{0.25,0.50,0.50}{##1}}}
\expandafter\def\csname PY@tok@cpf\endcsname{\let\PY@it=\textit\def\PY@tc##1{\textcolor[rgb]{0.25,0.50,0.50}{##1}}}
\expandafter\def\csname PY@tok@c1\endcsname{\let\PY@it=\textit\def\PY@tc##1{\textcolor[rgb]{0.25,0.50,0.50}{##1}}}
\expandafter\def\csname PY@tok@cs\endcsname{\let\PY@it=\textit\def\PY@tc##1{\textcolor[rgb]{0.25,0.50,0.50}{##1}}}

\def\PYZbs{\char`\\}
\def\PYZus{\char`\_}
\def\PYZob{\char`\{}
\def\PYZcb{\char`\}}
\def\PYZca{\char`\^}
\def\PYZam{\char`\&}
\def\PYZlt{\char`\<}
\def\PYZgt{\char`\>}
\def\PYZsh{\char`\#}
\def\PYZpc{\char`\%}
\def\PYZdl{\char`\$}
\def\PYZhy{\char`\-}
\def\PYZsq{\char`\'}
\def\PYZdq{\char`\"}
\def\PYZti{\char`\~}
% for compatibility with earlier versions
\def\PYZat{@}
\def\PYZlb{[}
\def\PYZrb{]}
\makeatother


    % Exact colors from NB
    \definecolor{incolor}{rgb}{0.0, 0.0, 0.5}
    \definecolor{outcolor}{rgb}{0.545, 0.0, 0.0}



    
    % Prevent overflowing lines due to hard-to-break entities
    \sloppy 
    % Setup hyperref package
    \hypersetup{
      breaklinks=true,  % so long urls are correctly broken across lines
      colorlinks=true,
      urlcolor=urlcolor,
      linkcolor=linkcolor,
      citecolor=citecolor,
      }
    % Slightly bigger margins than the latex defaults
    
    \geometry{verbose,tmargin=1in,bmargin=1in,lmargin=1in,rmargin=1in}
    
    

    \begin{document}
    
    
    \maketitle
    
    

    
    \section{ML- PS3, Juan Arroyo
Miranda}\label{ml--ps3-juan-arroyo-miranda}

    \begin{Verbatim}[commandchars=\\\{\}]
{\color{incolor}In [{\color{incolor}1}]:} \PY{o}{\PYZpc{}}\PY{k}{matplotlib} inline
        \PY{o}{\PYZpc{}}\PY{k}{run} classify
        \PY{n}{sns}\PY{o}{.}\PY{n}{set\PYZus{}style}\PY{p}{(}\PY{l+s+s1}{\PYZsq{}}\PY{l+s+s1}{whitegrid}\PY{l+s+s1}{\PYZsq{}}\PY{p}{)}
\end{Verbatim}


    \begin{Verbatim}[commandchars=\\\{\}]
/usr/local/lib/python3.4/dist-packages/IPython/html.py:14: ShimWarning: The `IPython.html` package has been deprecated. You should import from `notebook` instead. `IPython.html.widgets` has moved to `ipywidgets`.
  "`IPython.html.widgets` has moved to `ipywidgets`.", ShimWarning)

    \end{Verbatim}

    \section{Part 1, Fixing Processing of
Data}\label{part-1-fixing-processing-of-data}

    \begin{Verbatim}[commandchars=\\\{\}]
{\color{incolor}In [{\color{incolor}2}]:} \PY{n}{df} \PY{o}{=} \PY{n}{read\PYZus{}files}\PY{o}{.}\PY{n}{read\PYZus{}data}\PY{p}{(}\PY{l+s+s1}{\PYZsq{}}\PY{l+s+s1}{credit\PYZhy{}data.csv}\PY{l+s+s1}{\PYZsq{}}\PY{p}{)}
\end{Verbatim}


    \subsection{Data summary}\label{data-summary}

\subsubsection{In order to understand what factors make individuals more
likely to not pay debt, I computed the following summary statistics for
the data set: mean, mode, and missing
data}\label{in-order-to-understand-what-factors-make-individuals-more-likely-to-not-pay-debt-i-computed-the-following-summary-statistics-for-the-data-set-mean-mode-and-missing-data}

    \begin{Verbatim}[commandchars=\\\{\}]
{\color{incolor}In [{\color{incolor}3}]:} \PY{n}{df}\PY{o}{.}\PY{n}{mean}\PY{p}{(}\PY{p}{)}
\end{Verbatim}


\begin{Verbatim}[commandchars=\\\{\}]
{\color{outcolor}Out[{\color{outcolor}3}]:} personid                                75000.500000
        seriousdlqin2yrs                            0.066840
        revolvingutilizationofunsecuredlines        6.048438
        age                                        52.295207
        zipcode                                 60648.810013
        numberoftime30-59dayspastduenotworse        0.421033
        debtratio                                 353.005076
        monthlyincome                            6670.221237
        numberofopencreditlinesandloans             8.452760
        numberoftimes90dayslate                     0.265973
        numberrealestateloansorlines                1.018240
        numberoftime60-89dayspastduenotworse        0.240387
        numberofdependents                          0.757222
        dtype: float64
\end{Verbatim}
            
    \begin{Verbatim}[commandchars=\\\{\}]
{\color{incolor}In [{\color{incolor}4}]:} \PY{n}{df}\PY{o}{.}\PY{n}{mode}\PY{p}{(}\PY{p}{)}\PY{o}{.}\PY{n}{unstack}\PY{p}{(}\PY{p}{)}
\end{Verbatim}


\begin{Verbatim}[commandchars=\\\{\}]
{\color{outcolor}Out[{\color{outcolor}4}]:} personid                              0        NaN
        seriousdlqin2yrs                      0        0.0
        revolvingutilizationofunsecuredlines  0        0.0
        age                                   0       49.0
        zipcode                               0    60625.0
        numberoftime30-59dayspastduenotworse  0        0.0
        debtratio                             0        0.0
        monthlyincome                         0     5000.0
        numberofopencreditlinesandloans       0        6.0
        numberoftimes90dayslate               0        0.0
        numberrealestateloansorlines          0        0.0
        numberoftime60-89dayspastduenotworse  0        0.0
        numberofdependents                    0        0.0
        dtype: float64
\end{Verbatim}
            
    \subsubsection{Two variables have missing values that will be necessary
to impute: monthly income and number of
dependents.}\label{two-variables-have-missing-values-that-will-be-necessary-to-impute-monthly-income-and-number-of-dependents.}

    \begin{Verbatim}[commandchars=\\\{\}]
{\color{incolor}In [{\color{incolor}5}]:} \PY{n}{process\PYZus{}data}\PY{o}{.}\PY{n}{percentage\PYZus{}missing}\PY{p}{(}\PY{n}{df}\PY{p}{)}
\end{Verbatim}


    \begin{Verbatim}[commandchars=\\\{\}]
19.8\% missing from: Column monthlyincome
2.6\% missing from: Column numberofdependents

    \end{Verbatim}

    \subsection{Analysis for Number of
Dependents}\label{analysis-for-number-of-dependents}

To decide how to impute the data for the number of dependents I created
a function that does two things. First, it calculates the cumulative sum
of the value counts for each possible group in number of dependents.
Second, it computes the percentage of the observations that fall in each
one of these groups.

    \begin{Verbatim}[commandchars=\\\{\}]
{\color{incolor}In [{\color{incolor}6}]:} \PY{n}{process\PYZus{}data}\PY{o}{.}\PY{n}{des\PYZus{}num\PYZus{}dep}\PY{p}{(}\PY{n}{df}\PY{p}{)}
\end{Verbatim}


\begin{Verbatim}[commandchars=\\\{\}]
{\color{outcolor}Out[{\color{outcolor}6}]:}       numberofdependents  Cumulative\_Sum  Percentage\_By\_Group
        0.0                86902           86902            59.490950
        1.0                26316          113218            77.506230
        2.0                19522          132740            90.870506
        3.0                 9483          142223            97.362332
        4.0                 2862          145085            99.321586
        5.0                  746          145831            99.832279
        6.0                  158          145989            99.940442
        7.0                   51          146040            99.975355
        8.0                   24          146064            99.991785
        9.0                    5          146069            99.995208
        10.0                   5          146074            99.998631
        13.0                   1          146075            99.999315
        20.0                   1          146076           100.000000
\end{Verbatim}
            
    \subsection{Tables are nice, but some visualization is in
order}\label{tables-are-nice-but-some-visualization-is-in-order}

    \begin{Verbatim}[commandchars=\\\{\}]
{\color{incolor}In [{\color{incolor}7}]:} \PY{n}{process\PYZus{}data}\PY{o}{.}\PY{n}{frequency\PYZus{}plots}\PY{p}{(}\PY{n}{df}\PY{p}{,} \PY{l+s+s1}{\PYZsq{}}\PY{l+s+s1}{numberofdependents}\PY{l+s+s1}{\PYZsq{}}\PY{p}{)}
\end{Verbatim}


    \begin{center}
    \adjustimage{max size={0.9\linewidth}{0.9\paperheight}}{output_12_0.png}
    \end{center}
    { \hspace*{\fill} \\}
    
    The plots show that the distribution of the number of dependents is
skewed to the right, with most people (86,902) having zero dependents.
The boxplot shows the presence of some outliers, however, 90.87 \% of
the population concentrates between 0 and 2 dependents. Therefore, it
would be reasonable to impute dependents drawn randomly between 0 and 2.
While performing this imputation, I will take into account the
frequencies observed in the data for groups 0, 1, and 2.

    \subsection{Analysis for Monthly
Income}\label{analysis-for-monthly-income}

    \begin{Verbatim}[commandchars=\\\{\}]
{\color{incolor}In [{\color{incolor}8}]:} \PY{n+nb}{print}\PY{p}{(}\PY{n}{df}\PY{p}{[}\PY{l+s+s1}{\PYZsq{}}\PY{l+s+s1}{monthlyincome}\PY{l+s+s1}{\PYZsq{}}\PY{p}{]}\PY{o}{.}\PY{n}{describe}\PY{p}{(}\PY{n}{include} \PY{o}{=} \PY{l+s+s1}{\PYZsq{}}\PY{l+s+s1}{all}\PY{l+s+s1}{\PYZsq{}}\PY{p}{)}\PY{p}{)}
\end{Verbatim}


    \begin{Verbatim}[commandchars=\\\{\}]
count    1.202690e+05
mean     6.670221e+03
std      1.438467e+04
min      0.000000e+00
25\%      3.400000e+03
50\%      5.400000e+03
75\%      8.249000e+03
max      3.008750e+06
Name: monthlyincome, dtype: float64

    \end{Verbatim}

    This general description of the monthly income variable provides some
insights about the distribution of the data. The first thing that caught
my attention was the jump between the third quartile (\$ 8,249) and the
maximum (\$3,008,750). I decided to take a look at the deciles to gain a
better understanding of this variable

    \subsubsection{Pandas default method for computing percentiles does not
take Nan values into
account!!!}\label{pandas-default-method-for-computing-percentiles-does-not-take-nan-values-into-account}

    Using numpy's percentile function made me realize that I was getting
different values from the ones computed by pandas' describe function.
There are two interesting results from computing the deciles for monthly
income: 1. The 80th percentile changes from \$9,083 using nanpercentile
function to \$54,166 using a function that takes into account the Nan
values. 2. Most Nan observations come from the highest percentiles in
the income distribution

    \begin{Verbatim}[commandchars=\\\{\}]
{\color{incolor}In [{\color{incolor}9}]:} \PY{c+c1}{\PYZsh{} Income deciles is a better description of the data (ignoring nan)}
        \PY{k}{for} \PY{n}{i} \PY{o+ow}{in} \PY{n+nb}{range}\PY{p}{(}\PY{l+m+mi}{10}\PY{p}{,}\PY{l+m+mi}{101}\PY{p}{,}\PY{l+m+mi}{5}\PY{p}{)}\PY{p}{:}
         \PY{n+nb}{print}\PY{p}{(}\PY{n}{i}\PY{p}{,} \PY{l+s+s1}{\PYZsq{}}\PY{l+s+s1}{Nan percentile}\PY{l+s+s1}{\PYZsq{}}\PY{p}{,} \PY{n}{np}\PY{o}{.}\PY{n}{nanpercentile}\PY{p}{(}\PY{n}{df}\PY{p}{[}\PY{l+s+s1}{\PYZsq{}}\PY{l+s+s1}{monthlyincome}\PY{l+s+s1}{\PYZsq{}}\PY{p}{]}\PY{p}{,} \PY{n}{i}\PY{p}{)}\PY{p}{,} \PY{l+s+s2}{\PYZdq{}}\PY{l+s+s2}{|}\PY{l+s+s2}{\PYZdq{}}\PY{p}{,} \PY{l+s+s1}{\PYZsq{}}\PY{l+s+s1}{percentile}\PY{l+s+s1}{\PYZsq{}}\PY{p}{,} \PY{n}{np}\PY{o}{.}\PY{n}{percentile}\PY{p}{(}\PY{n}{df}\PY{p}{[}\PY{l+s+s1}{\PYZsq{}}\PY{l+s+s1}{monthlyincome}\PY{l+s+s1}{\PYZsq{}}\PY{p}{]}\PY{p}{,} \PY{n}{i}\PY{p}{)} \PY{p}{)}
\end{Verbatim}


    \begin{Verbatim}[commandchars=\\\{\}]
10 Nan percentile 2005.0 | percentile 2325.0
15 Nan percentile 2500.0 | percentile 2904.0
20 Nan percentile 3000.0 | percentile 3400.0
25 Nan percentile 3400.0 | percentile 3903.0
30 Nan percentile 3800.0 | percentile 4333.0
35 Nan percentile 4166.0 | percentile 4906.0
40 Nan percentile 4544.2 | percentile 5400.0
45 Nan percentile 5000.0 | percentile 6000.0
50 Nan percentile 5400.0 | percentile 6600.0
55 Nan percentile 5855.0 | percentile 7339.0
60 Nan percentile 6300.0 | percentile 8200.0
65 Nan percentile 6916.0 | percentile 9318.0
70 Nan percentile 7500.0 | percentile 10660.0
75 Nan percentile 8249.0 | percentile 13333.0
80 Nan percentile 9083.0 | percentile 54166.0
85 Nan percentile 10100.0 | percentile nan
90 Nan percentile 11666.0 | percentile nan
95 Nan percentile 14587.6 | percentile nan
100 Nan percentile 3008750.0 | percentile nan

    \end{Verbatim}

    \begin{Verbatim}[commandchars=\\\{\}]
{\color{incolor}In [{\color{incolor}10}]:} \PY{n}{process\PYZus{}data}\PY{o}{.}\PY{n}{plot\PYZus{}income\PYZus{}distribution}\PY{p}{(}\PY{n}{df}\PY{p}{,} \PY{l+s+s1}{\PYZsq{}}\PY{l+s+s1}{monthlyincome}\PY{l+s+s1}{\PYZsq{}}\PY{p}{)}
\end{Verbatim}


    \begin{center}
    \adjustimage{max size={0.9\linewidth}{0.9\paperheight}}{output_20_0.png}
    \end{center}
    { \hspace*{\fill} \\}
    
    As expected, the monthly income distribution is asymmetric. We can
observe this once we divide the distribution by deciles. The first plot
(right top corner) shows the income distribution up to the 80th
percentile, from this plot, it is clear that most people earn less than
\$10,000 per month. Above the 80th percentile and below \$1,000,000, we
observe the same pattern with most people earning around \$100,000 per
month. It is interesting to note that the percentile function returns
nan (versus nanpercentile function) for percentiles equal to and above
the 85th percentile. This might suggest that the richest people in the
sample tend not to report their monthly income.

Since we have 29,731 missing values, it might safe to assume that they
come from the average population and not from the outliers. Therefore, I
decided to impute the data with the mean value.

    \subsection{Splitting Data and doing
imputation}\label{splitting-data-and-doing-imputation}

    \begin{Verbatim}[commandchars=\\\{\}]
{\color{incolor}In [{\color{incolor}11}]:} \PY{c+c1}{\PYZsh{} Data imputation for Number of Dependents}
         \PY{n}{process\PYZus{}data}\PY{o}{.}\PY{n}{impute\PYZus{}val\PYZus{}to\PYZus{}column}\PY{p}{(}\PY{n}{df}\PY{p}{,}\PY{l+s+s1}{\PYZsq{}}\PY{l+s+s1}{numberofdependents}\PY{l+s+s1}{\PYZsq{}}\PY{p}{,} \PY{l+s+s1}{\PYZsq{}}\PY{l+s+s1}{random}\PY{l+s+s1}{\PYZsq{}}\PY{p}{,} \PY{l+m+mi}{0}\PY{p}{,} \PY{l+m+mi}{3}\PY{p}{,} \PY{p}{[}\PY{l+m+mf}{0.65}\PY{p}{,}\PY{l+m+mf}{0.2}\PY{p}{,}\PY{l+m+mf}{0.15}\PY{p}{]}\PY{p}{)}
\end{Verbatim}


    \begin{Verbatim}[commandchars=\\\{\}]
{\color{incolor}In [{\color{incolor}12}]:} \PY{c+c1}{\PYZsh{} Impute income data}
         \PY{n}{process\PYZus{}data}\PY{o}{.}\PY{n}{impute\PYZus{}val\PYZus{}to\PYZus{}column}\PY{p}{(}\PY{n}{df}\PY{p}{,}\PY{l+s+s1}{\PYZsq{}}\PY{l+s+s1}{monthlyincome}\PY{l+s+s1}{\PYZsq{}}\PY{p}{,} \PY{l+s+s1}{\PYZsq{}}\PY{l+s+s1}{mean}\PY{l+s+s1}{\PYZsq{}}\PY{p}{)}
\end{Verbatim}


    \begin{Verbatim}[commandchars=\\\{\}]
{\color{incolor}In [{\color{incolor}13}]:} \PY{n}{df}\PY{p}{[}\PY{l+s+s1}{\PYZsq{}}\PY{l+s+s1}{age}\PY{l+s+s1}{\PYZsq{}}\PY{p}{]}\PY{o}{.}\PY{n}{replace}\PY{p}{(}\PY{l+m+mi}{0}\PY{p}{,} \PY{n}{df}\PY{p}{[}\PY{l+s+s1}{\PYZsq{}}\PY{l+s+s1}{age}\PY{l+s+s1}{\PYZsq{}}\PY{p}{]}\PY{o}{.}\PY{n}{mean}\PY{p}{(}\PY{p}{)}\PY{p}{,} \PY{n}{inplace} \PY{o}{=} \PY{k+kc}{True}\PY{p}{)}
\end{Verbatim}


    \subsubsection{Check imputation was done
correctly}\label{check-imputation-was-done-correctly}

    \begin{Verbatim}[commandchars=\\\{\}]
{\color{incolor}In [{\color{incolor}14}]:} \PY{n}{df}\PY{o}{.}\PY{n}{isnull}\PY{p}{(}\PY{p}{)}\PY{o}{.}\PY{n}{sum}\PY{p}{(}\PY{p}{)}
\end{Verbatim}


\begin{Verbatim}[commandchars=\\\{\}]
{\color{outcolor}Out[{\color{outcolor}14}]:} personid                                0
         seriousdlqin2yrs                        0
         revolvingutilizationofunsecuredlines    0
         age                                     0
         zipcode                                 0
         numberoftime30-59dayspastduenotworse    0
         debtratio                               0
         monthlyincome                           0
         numberofopencreditlinesandloans         0
         numberoftimes90dayslate                 0
         numberrealestateloansorlines            0
         numberoftime60-89dayspastduenotworse    0
         numberofdependents                      0
         dtype: int64
\end{Verbatim}
            
    \subsection{Discretize and Visualize data by
buckets}\label{discretize-and-visualize-data-by-buckets}

    \begin{Verbatim}[commandchars=\\\{\}]
{\color{incolor}In [{\color{incolor}15}]:} \PY{n}{bins\PYZus{}income} \PY{o}{=} \PY{n+nb}{range}\PY{p}{(}\PY{l+m+mi}{0}\PY{p}{,} \PY{l+m+mi}{100000}\PY{p}{,} \PY{l+m+mi}{5000}\PY{p}{)}
         \PY{n}{bins\PYZus{}age} \PY{o}{=} \PY{n+nb}{range}\PY{p}{(}\PY{l+m+mi}{20}\PY{p}{,}\PY{l+m+mi}{120}\PY{p}{,}\PY{l+m+mi}{5}\PY{p}{)}
         \PY{n}{income\PYZus{}bucket} \PY{o}{=} \PY{n}{process\PYZus{}data}\PY{o}{.}\PY{n}{discretize}\PY{p}{(}\PY{n}{df}\PY{p}{,} \PY{l+s+s1}{\PYZsq{}}\PY{l+s+s1}{monthlyincome}\PY{l+s+s1}{\PYZsq{}}\PY{p}{,} \PY{n}{bins\PYZus{}income}\PY{p}{)}
         \PY{n}{age\PYZus{}bucket} \PY{o}{=} \PY{n}{process\PYZus{}data}\PY{o}{.}\PY{n}{discretize}\PY{p}{(}\PY{n}{df}\PY{p}{,} \PY{l+s+s1}{\PYZsq{}}\PY{l+s+s1}{age}\PY{l+s+s1}{\PYZsq{}}\PY{p}{,} \PY{n}{bins\PYZus{}age}\PY{p}{)}
\end{Verbatim}


    \subsubsection{In the following subplots, I explore some relationships
that seem intuitive at first glance. For example, we expect that
Seriousdlqin2yrs is positively correlated with the number of dependents,
since having more children may difficult the payment of
debts.}\label{in-the-following-subplots-i-explore-some-relationships-that-seem-intuitive-at-first-glance.-for-example-we-expect-that-seriousdlqin2yrs-is-positively-correlated-with-the-number-of-dependents-since-having-more-children-may-difficult-the-payment-of-debts.}

\subsubsection{By dividing the mothly income into buckets, we
incorporate our findings from the percentile distribution. Since most
observations are within the 0 and 100, 000 usd, I chose not to
incorporate outliers in this graph. It is interesting to note that the
highest financial distress in for people with an income of
\$100,000.}\label{by-dividing-the-mothly-income-into-buckets-we-incorporate-our-findings-from-the-percentile-distribution.-since-most-observations-are-within-the-0-and-100-000-usd-i-chose-not-to-incorporate-outliers-in-this-graph.-it-is-interesting-to-note-that-the-highest-financial-distress-in-for-people-with-an-income-of-100000.}

\subsubsection{Two curious facts abouts the age
buckets:}\label{two-curious-facts-abouts-the-age-buckets}

\subsubsection{1. People between 25 and 30 years are more likely not to
pay their debts, also people between 95 and
100.}\label{people-between-25-and-30-years-are-more-likely-not-to-pay-their-debts-also-people-between-95-and-100.}

\subsubsection{2. The highest debt to income rage is for 95 - 100
group.}\label{the-highest-debt-to-income-rage-is-for-95---100-group.}

    \begin{Verbatim}[commandchars=\\\{\}]
{\color{incolor}In [{\color{incolor}16}]:} \PY{n}{process\PYZus{}data}\PY{o}{.}\PY{n}{visualize\PYZus{}buckets}\PY{p}{(}\PY{n}{df}\PY{p}{,} \PY{l+s+s1}{\PYZsq{}}\PY{l+s+s1}{numberofdependents}\PY{l+s+s1}{\PYZsq{}}\PY{p}{,} \PY{l+s+s1}{\PYZsq{}}\PY{l+s+s1}{seriousdlqin2yrs}\PY{l+s+s1}{\PYZsq{}}\PY{p}{,} \PY{l+s+s1}{\PYZsq{}}\PY{l+s+s1}{debtratio}\PY{l+s+s1}{\PYZsq{}}\PY{p}{,} \PY{l+s+s1}{\PYZsq{}}\PY{l+s+s1}{bins\PYZus{}age}\PY{l+s+s1}{\PYZsq{}}\PY{p}{,} \PY{l+s+s1}{\PYZsq{}}\PY{l+s+s1}{bins\PYZus{}monthlyincome}\PY{l+s+s1}{\PYZsq{}}\PY{p}{)}
\end{Verbatim}


    \begin{center}
    \adjustimage{max size={0.9\linewidth}{0.9\paperheight}}{output_31_0.png}
    \end{center}
    { \hspace*{\fill} \\}
    
    Looking at the income graph, it is easy to observe a high variability in
the data. This variability will 'contaminate' with noise our prediction.
Normalization is one way to handle this issue.

    \begin{Verbatim}[commandchars=\\\{\}]
{\color{incolor}In [{\color{incolor}17}]:} \PY{n}{process\PYZus{}data}\PY{o}{.}\PY{n}{plot\PYZus{}corr\PYZus{}matrix}\PY{p}{(}\PY{n}{df}\PY{p}{)}
\end{Verbatim}


    \begin{center}
    \adjustimage{max size={0.9\linewidth}{0.9\paperheight}}{output_33_0.png}
    \end{center}
    { \hspace*{\fill} \\}
    
    \subsection{Categorical to Dummy}\label{categorical-to-dummy}

    \begin{Verbatim}[commandchars=\\\{\}]
{\color{incolor}In [{\color{incolor}18}]:} \PY{n}{df}\PY{p}{[}\PY{l+s+s1}{\PYZsq{}}\PY{l+s+s1}{zipcode}\PY{l+s+s1}{\PYZsq{}}\PY{p}{]}\PY{o}{.}\PY{n}{astype}\PY{p}{(}\PY{l+s+s2}{\PYZdq{}}\PY{l+s+s2}{category}\PY{l+s+s2}{\PYZdq{}}\PY{p}{)}
         \PY{n}{process\PYZus{}data}\PY{o}{.}\PY{n}{dummify}\PY{p}{(}\PY{n}{df}\PY{p}{,} \PY{l+s+s1}{\PYZsq{}}\PY{l+s+s1}{zipcode}\PY{l+s+s1}{\PYZsq{}}\PY{p}{)}\PY{o}{.}\PY{n}{tail}\PY{p}{(}\PY{p}{)}
\end{Verbatim}


\begin{Verbatim}[commandchars=\\\{\}]
{\color{outcolor}Out[{\color{outcolor}18}]:}         personid  seriousdlqin2yrs  revolvingutilizationofunsecuredlines  \textbackslash{}
         149995    149996                 0                              0.040674   
         149996    149997                 0                              0.299745   
         149997    149998                 0                              0.246044   
         149998    149999                 0                              0.000000   
         149999    150000                 0                              0.850283   
         
                  age  zipcode  numberoftime30-59dayspastduenotworse    debtratio  \textbackslash{}
         149995  74.0    60637                                     0     0.225131   
         149996  44.0    60625                                     0     0.716562   
         149997  58.0    60625                                     0  3870.000000   
         149998  30.0    60625                                     0     0.000000   
         149999  64.0    60657                                     0     0.249908   
         
                 monthlyincome  numberofopencreditlinesandloans  \textbackslash{}
         149995    2100.000000                                4   
         149996    5584.000000                                4   
         149997    6670.221237                               18   
         149998    5716.000000                                4   
         149999    8158.000000                                8   
         
                 numberoftimes90dayslate  {\ldots}    bins\_monthlyincome  bins\_age  60601  \textbackslash{}
         149995                        0  {\ldots}             (0, 5000]  (70, 75]      0   
         149996                        0  {\ldots}         (5000, 10000]  (40, 45]      0   
         149997                        0  {\ldots}         (5000, 10000]  (55, 60]      0   
         149998                        0  {\ldots}         (5000, 10000]  (25, 30]      0   
         149999                        0  {\ldots}         (5000, 10000]  (60, 65]      0   
         
                60618 60625  60629  60637  60644  60657  60804  
         149995     0     0      0      1      0      0      0  
         149996     0     1      0      0      0      0      0  
         149997     0     1      0      0      0      0      0  
         149998     0     1      0      0      0      0      0  
         149999     0     0      0      0      0      1      0  
         
         [5 rows x 23 columns]
\end{Verbatim}
            
    \begin{Verbatim}[commandchars=\\\{\}]
{\color{incolor}In [{\color{incolor}19}]:} \PY{n}{df} \PY{o}{=} \PY{n}{df}\PY{o}{.}\PY{n}{drop}\PY{p}{(}\PY{p}{[}\PY{l+s+s1}{\PYZsq{}}\PY{l+s+s1}{bins\PYZus{}age}\PY{l+s+s1}{\PYZsq{}}\PY{p}{,} \PY{l+s+s1}{\PYZsq{}}\PY{l+s+s1}{bins\PYZus{}monthlyincome}\PY{l+s+s1}{\PYZsq{}}\PY{p}{]}\PY{p}{,} \PY{n}{axis} \PY{o}{=}\PY{l+m+mi}{1}\PY{p}{)}
\end{Verbatim}


    \section{Parts 2 - 4, Model Selection and
Implementation}\label{parts-2---4-model-selection-and-implementation}

    \subsubsection{Normalize data with high
variance}\label{normalize-data-with-high-variance}

    \begin{Verbatim}[commandchars=\\\{\}]
{\color{incolor}In [{\color{incolor}20}]:} \PY{c+c1}{\PYZsh{} Normalize income data }
         \PY{n}{df}\PY{p}{[}\PY{l+s+s1}{\PYZsq{}}\PY{l+s+s1}{monthlyincome}\PY{l+s+s1}{\PYZsq{}}\PY{p}{]} \PY{o}{=} \PY{n}{StandardScaler}\PY{p}{(}\PY{p}{)}\PY{o}{.}\PY{n}{fit\PYZus{}transform}\PY{p}{(}\PY{n}{df}\PY{p}{[}\PY{l+s+s1}{\PYZsq{}}\PY{l+s+s1}{monthlyincome}\PY{l+s+s1}{\PYZsq{}}\PY{p}{]}\PY{p}{)}
         \PY{c+c1}{\PYZsh{} Normalize revolving utilization of unsecured lines}
         \PY{n}{df}\PY{p}{[}\PY{l+s+s1}{\PYZsq{}}\PY{l+s+s1}{revolvingutilizationofunsecuredlines}\PY{l+s+s1}{\PYZsq{}}\PY{p}{]} \PY{o}{=} \PY{n}{StandardScaler}\PY{p}{(}\PY{p}{)}\PY{o}{.}\PY{n}{fit\PYZus{}transform}\PY{p}{(}\PY{n}{df}\PY{p}{[}\PY{l+s+s1}{\PYZsq{}}\PY{l+s+s1}{revolvingutilizationofunsecuredlines}\PY{l+s+s1}{\PYZsq{}}\PY{p}{]}\PY{p}{)}
\end{Verbatim}


    \begin{Verbatim}[commandchars=\\\{\}]
{\color{incolor}In [{\color{incolor}21}]:} \PY{n}{y} \PY{o}{=} \PY{n}{df}\PY{p}{[}\PY{l+s+s1}{\PYZsq{}}\PY{l+s+s1}{seriousdlqin2yrs}\PY{l+s+s1}{\PYZsq{}}\PY{p}{]}\PY{o}{.}\PY{n}{values}
         \PY{n}{X} \PY{o}{=} \PY{n}{df}\PY{o}{.}\PY{n}{drop}\PY{p}{(}\PY{l+s+s1}{\PYZsq{}}\PY{l+s+s1}{seriousdlqin2yrs}\PY{l+s+s1}{\PYZsq{}}\PY{p}{,} \PY{n}{axis}\PY{o}{=}\PY{l+m+mi}{1}\PY{p}{)}\PY{o}{.}\PY{n}{values}
\end{Verbatim}


    \subsection{Part 2: Adding Classifiers}\label{part-2-adding-classifiers}

    \subsubsection{In the following part, I call the do\_learning function
which implements a modified version of Rayid Ghani's magic loop function
to run different classifiers and parameter
combinations.}\label{in-the-following-part-i-call-the-do_learning-function-which-implements-a-modified-version-of-rayid-ghanis-magic-loop-function-to-run-different-classifiers-and-parameter-combinations.}

\subsubsection{The models to run are : Logistic Regression, Knn,
Decision Tree, Random Forests, Boosting, Bagging, and Linear
SVM}\label{the-models-to-run-are-logistic-regression-knn-decision-tree-random-forests-boosting-bagging-and-linear-svm}

    \begin{Verbatim}[commandchars=\\\{\}]
{\color{incolor}In [{\color{incolor}22}]:} \PY{n}{features}  \PY{o}{=}  \PY{p}{[}\PY{l+s+s1}{\PYZsq{}}\PY{l+s+s1}{numberoftime30\PYZhy{}59dayspastduenotworse}\PY{l+s+s1}{\PYZsq{}}\PY{p}{,} 
                       \PY{l+s+s1}{\PYZsq{}}\PY{l+s+s1}{monthlyincome}\PY{l+s+s1}{\PYZsq{}}\PY{p}{,} \PY{l+s+s1}{\PYZsq{}}\PY{l+s+s1}{numberrealestateloansorlines}\PY{l+s+s1}{\PYZsq{}}\PY{p}{,} \PY{l+s+s1}{\PYZsq{}}\PY{l+s+s1}{numberofdependents}\PY{l+s+s1}{\PYZsq{}}\PY{p}{]}
         
         \PY{n}{models\PYZus{}to\PYZus{}run}\PY{o}{=}\PY{p}{[}\PY{l+s+s1}{\PYZsq{}}\PY{l+s+s1}{LR}\PY{l+s+s1}{\PYZsq{}}\PY{p}{,}\PY{l+s+s1}{\PYZsq{}}\PY{l+s+s1}{KNN}\PY{l+s+s1}{\PYZsq{}}\PY{p}{,}\PY{l+s+s1}{\PYZsq{}}\PY{l+s+s1}{DT}\PY{l+s+s1}{\PYZsq{}}\PY{p}{,} \PY{l+s+s1}{\PYZsq{}}\PY{l+s+s1}{RF}\PY{l+s+s1}{\PYZsq{}}\PY{p}{,} \PY{l+s+s1}{\PYZsq{}}\PY{l+s+s1}{AB}\PY{l+s+s1}{\PYZsq{}}\PY{p}{,}\PY{l+s+s1}{\PYZsq{}}\PY{l+s+s1}{GB}\PY{l+s+s1}{\PYZsq{}}\PY{p}{,}\PY{l+s+s1}{\PYZsq{}}\PY{l+s+s1}{LSVC}\PY{l+s+s1}{\PYZsq{}}\PY{p}{]}
\end{Verbatim}


    \subsection{Part 3: Defining Parameter
Grids}\label{part-3-defining-parameter-grids}

    \begin{Verbatim}[commandchars=\\\{\}]
{\color{incolor}In [{\color{incolor}23}]:}     \PY{n}{test\PYZus{}grid} \PY{o}{=} \PY{p}{\PYZob{}} 
             \PY{l+s+s1}{\PYZsq{}}\PY{l+s+s1}{RF}\PY{l+s+s1}{\PYZsq{}}\PY{p}{:}\PY{p}{\PYZob{}}\PY{l+s+s1}{\PYZsq{}}\PY{l+s+s1}{n\PYZus{}estimators}\PY{l+s+s1}{\PYZsq{}}\PY{p}{:} \PY{p}{[}\PY{l+m+mi}{1}\PY{p}{]}\PY{p}{,} \PY{l+s+s1}{\PYZsq{}}\PY{l+s+s1}{max\PYZus{}depth}\PY{l+s+s1}{\PYZsq{}}\PY{p}{:} \PY{p}{[}\PY{l+m+mi}{1}\PY{p}{]}\PY{p}{,} \PY{l+s+s1}{\PYZsq{}}\PY{l+s+s1}{max\PYZus{}features}\PY{l+s+s1}{\PYZsq{}}\PY{p}{:} \PY{p}{[}\PY{l+s+s1}{\PYZsq{}}\PY{l+s+s1}{sqrt}\PY{l+s+s1}{\PYZsq{}}\PY{p}{]}\PY{p}{,}\PY{l+s+s1}{\PYZsq{}}\PY{l+s+s1}{min\PYZus{}samples\PYZus{}split}\PY{l+s+s1}{\PYZsq{}}\PY{p}{:} \PY{p}{[}\PY{l+m+mi}{10}\PY{p}{]}\PY{p}{\PYZcb{}}\PY{p}{,}
             \PY{l+s+s1}{\PYZsq{}}\PY{l+s+s1}{LR}\PY{l+s+s1}{\PYZsq{}}\PY{p}{:} \PY{p}{\PYZob{}} \PY{l+s+s1}{\PYZsq{}}\PY{l+s+s1}{penalty}\PY{l+s+s1}{\PYZsq{}}\PY{p}{:} \PY{p}{[}\PY{l+s+s1}{\PYZsq{}}\PY{l+s+s1}{l1}\PY{l+s+s1}{\PYZsq{}}\PY{p}{]}\PY{p}{,} \PY{l+s+s1}{\PYZsq{}}\PY{l+s+s1}{C}\PY{l+s+s1}{\PYZsq{}}\PY{p}{:} \PY{p}{[}\PY{l+m+mf}{0.01}\PY{p}{]}\PY{p}{\PYZcb{}}\PY{p}{,}
             \PY{l+s+s1}{\PYZsq{}}\PY{l+s+s1}{ET}\PY{l+s+s1}{\PYZsq{}}\PY{p}{:} \PY{p}{\PYZob{}} \PY{l+s+s1}{\PYZsq{}}\PY{l+s+s1}{n\PYZus{}estimators}\PY{l+s+s1}{\PYZsq{}}\PY{p}{:} \PY{p}{[}\PY{l+m+mi}{1}\PY{p}{]}\PY{p}{,} \PY{l+s+s1}{\PYZsq{}}\PY{l+s+s1}{criterion}\PY{l+s+s1}{\PYZsq{}} \PY{p}{:} \PY{p}{[}\PY{l+s+s1}{\PYZsq{}}\PY{l+s+s1}{gini}\PY{l+s+s1}{\PYZsq{}}\PY{p}{]} \PY{p}{,}\PY{l+s+s1}{\PYZsq{}}\PY{l+s+s1}{max\PYZus{}depth}\PY{l+s+s1}{\PYZsq{}}\PY{p}{:} \PY{p}{[}\PY{l+m+mi}{1}\PY{p}{]}\PY{p}{,} \PY{l+s+s1}{\PYZsq{}}\PY{l+s+s1}{max\PYZus{}features}\PY{l+s+s1}{\PYZsq{}}\PY{p}{:} \PY{p}{[}\PY{l+s+s1}{\PYZsq{}}\PY{l+s+s1}{sqrt}\PY{l+s+s1}{\PYZsq{}}\PY{p}{]}\PY{p}{,}\PY{l+s+s1}{\PYZsq{}}\PY{l+s+s1}{min\PYZus{}samples\PYZus{}split}\PY{l+s+s1}{\PYZsq{}}\PY{p}{:} \PY{p}{[}\PY{l+m+mi}{10}\PY{p}{]}\PY{p}{\PYZcb{}}\PY{p}{,}
             \PY{l+s+s1}{\PYZsq{}}\PY{l+s+s1}{AB}\PY{l+s+s1}{\PYZsq{}}\PY{p}{:} \PY{p}{\PYZob{}} \PY{l+s+s1}{\PYZsq{}}\PY{l+s+s1}{algorithm}\PY{l+s+s1}{\PYZsq{}}\PY{p}{:} \PY{p}{[}\PY{l+s+s1}{\PYZsq{}}\PY{l+s+s1}{SAMME}\PY{l+s+s1}{\PYZsq{}}\PY{p}{]}\PY{p}{,} \PY{l+s+s1}{\PYZsq{}}\PY{l+s+s1}{n\PYZus{}estimators}\PY{l+s+s1}{\PYZsq{}}\PY{p}{:} \PY{p}{[}\PY{l+m+mi}{1}\PY{p}{]}\PY{p}{\PYZcb{}}\PY{p}{,}
             \PY{l+s+s1}{\PYZsq{}}\PY{l+s+s1}{GB}\PY{l+s+s1}{\PYZsq{}}\PY{p}{:} \PY{p}{\PYZob{}}\PY{l+s+s1}{\PYZsq{}}\PY{l+s+s1}{n\PYZus{}estimators}\PY{l+s+s1}{\PYZsq{}}\PY{p}{:} \PY{p}{[}\PY{l+m+mi}{1}\PY{p}{]}\PY{p}{,} \PY{l+s+s1}{\PYZsq{}}\PY{l+s+s1}{learning\PYZus{}rate}\PY{l+s+s1}{\PYZsq{}} \PY{p}{:} \PY{p}{[}\PY{l+m+mf}{0.1}\PY{p}{]}\PY{p}{,}\PY{l+s+s1}{\PYZsq{}}\PY{l+s+s1}{subsample}\PY{l+s+s1}{\PYZsq{}} \PY{p}{:} \PY{p}{[}\PY{l+m+mf}{0.5}\PY{p}{]}\PY{p}{,} \PY{l+s+s1}{\PYZsq{}}\PY{l+s+s1}{max\PYZus{}depth}\PY{l+s+s1}{\PYZsq{}}\PY{p}{:} \PY{p}{[}\PY{l+m+mi}{1}\PY{p}{]}\PY{p}{\PYZcb{}}\PY{p}{,}
             \PY{l+s+s1}{\PYZsq{}}\PY{l+s+s1}{NB}\PY{l+s+s1}{\PYZsq{}} \PY{p}{:} \PY{p}{\PYZob{}}\PY{p}{\PYZcb{}}\PY{p}{,}
             \PY{l+s+s1}{\PYZsq{}}\PY{l+s+s1}{DT}\PY{l+s+s1}{\PYZsq{}}\PY{p}{:} \PY{p}{\PYZob{}}\PY{l+s+s1}{\PYZsq{}}\PY{l+s+s1}{criterion}\PY{l+s+s1}{\PYZsq{}}\PY{p}{:} \PY{p}{[}\PY{l+s+s1}{\PYZsq{}}\PY{l+s+s1}{gini}\PY{l+s+s1}{\PYZsq{}}\PY{p}{]}\PY{p}{,} \PY{l+s+s1}{\PYZsq{}}\PY{l+s+s1}{max\PYZus{}depth}\PY{l+s+s1}{\PYZsq{}}\PY{p}{:} \PY{p}{[}\PY{l+m+mi}{1}\PY{p}{]}\PY{p}{,} \PY{l+s+s1}{\PYZsq{}}\PY{l+s+s1}{max\PYZus{}features}\PY{l+s+s1}{\PYZsq{}}\PY{p}{:} \PY{p}{[}\PY{l+s+s1}{\PYZsq{}}\PY{l+s+s1}{sqrt}\PY{l+s+s1}{\PYZsq{}}\PY{p}{]}\PY{p}{,}\PY{l+s+s1}{\PYZsq{}}\PY{l+s+s1}{min\PYZus{}samples\PYZus{}split}\PY{l+s+s1}{\PYZsq{}}\PY{p}{:} \PY{p}{[}\PY{l+m+mi}{10}\PY{p}{]}\PY{p}{\PYZcb{}}\PY{p}{,}
             \PY{l+s+s1}{\PYZsq{}}\PY{l+s+s1}{LSVC}\PY{l+s+s1}{\PYZsq{}} \PY{p}{:}\PY{p}{\PYZob{}}\PY{l+s+s1}{\PYZsq{}}\PY{l+s+s1}{C}\PY{l+s+s1}{\PYZsq{}} \PY{p}{:}\PY{p}{[}\PY{l+m+mf}{0.01}\PY{p}{]}\PY{p}{\PYZcb{}}\PY{p}{,}
             \PY{l+s+s1}{\PYZsq{}}\PY{l+s+s1}{KNN}\PY{l+s+s1}{\PYZsq{}} \PY{p}{:}\PY{p}{\PYZob{}}\PY{l+s+s1}{\PYZsq{}}\PY{l+s+s1}{n\PYZus{}neighbors}\PY{l+s+s1}{\PYZsq{}}\PY{p}{:} \PY{p}{[}\PY{l+m+mi}{5}\PY{p}{]}\PY{p}{,}\PY{l+s+s1}{\PYZsq{}}\PY{l+s+s1}{weights}\PY{l+s+s1}{\PYZsq{}}\PY{p}{:} \PY{p}{[}\PY{l+s+s1}{\PYZsq{}}\PY{l+s+s1}{uniform}\PY{l+s+s1}{\PYZsq{}}\PY{p}{]}\PY{p}{,}\PY{l+s+s1}{\PYZsq{}}\PY{l+s+s1}{algorithm}\PY{l+s+s1}{\PYZsq{}}\PY{p}{:} \PY{p}{[}\PY{l+s+s1}{\PYZsq{}}\PY{l+s+s1}{auto}\PY{l+s+s1}{\PYZsq{}}\PY{p}{]}\PY{p}{\PYZcb{}}
                    \PY{p}{\PYZcb{}}
\end{Verbatim}


    \begin{Verbatim}[commandchars=\\\{\}]
{\color{incolor}In [{\color{incolor}24}]:}     \PY{n}{small\PYZus{}grid} \PY{o}{=} \PY{p}{\PYZob{}} 
             \PY{l+s+s1}{\PYZsq{}}\PY{l+s+s1}{RF}\PY{l+s+s1}{\PYZsq{}}\PY{p}{:}\PY{p}{\PYZob{}}\PY{l+s+s1}{\PYZsq{}}\PY{l+s+s1}{n\PYZus{}estimators}\PY{l+s+s1}{\PYZsq{}}\PY{p}{:} \PY{p}{[}\PY{l+m+mi}{10}\PY{p}{,}\PY{l+m+mi}{100}\PY{p}{]}\PY{p}{,} \PY{l+s+s1}{\PYZsq{}}\PY{l+s+s1}{max\PYZus{}depth}\PY{l+s+s1}{\PYZsq{}}\PY{p}{:} \PY{p}{[}\PY{l+m+mi}{5}\PY{p}{,}\PY{l+m+mi}{50}\PY{p}{]}\PY{p}{,} \PY{l+s+s1}{\PYZsq{}}\PY{l+s+s1}{max\PYZus{}features}\PY{l+s+s1}{\PYZsq{}}\PY{p}{:} \PY{p}{[}\PY{l+s+s1}{\PYZsq{}}\PY{l+s+s1}{sqrt}\PY{l+s+s1}{\PYZsq{}}\PY{p}{,}\PY{l+s+s1}{\PYZsq{}}\PY{l+s+s1}{log2}\PY{l+s+s1}{\PYZsq{}}\PY{p}{]}\PY{p}{,}\PY{l+s+s1}{\PYZsq{}}\PY{l+s+s1}{min\PYZus{}samples\PYZus{}split}\PY{l+s+s1}{\PYZsq{}}\PY{p}{:} \PY{p}{[}\PY{l+m+mi}{2}\PY{p}{,}\PY{l+m+mi}{10}\PY{p}{]}\PY{p}{\PYZcb{}}\PY{p}{,}
             \PY{l+s+s1}{\PYZsq{}}\PY{l+s+s1}{LR}\PY{l+s+s1}{\PYZsq{}}\PY{p}{:} \PY{p}{\PYZob{}} \PY{l+s+s1}{\PYZsq{}}\PY{l+s+s1}{penalty}\PY{l+s+s1}{\PYZsq{}}\PY{p}{:} \PY{p}{[}\PY{l+s+s1}{\PYZsq{}}\PY{l+s+s1}{l1}\PY{l+s+s1}{\PYZsq{}}\PY{p}{,}\PY{l+s+s1}{\PYZsq{}}\PY{l+s+s1}{l2}\PY{l+s+s1}{\PYZsq{}}\PY{p}{]}\PY{p}{,} \PY{l+s+s1}{\PYZsq{}}\PY{l+s+s1}{C}\PY{l+s+s1}{\PYZsq{}}\PY{p}{:} \PY{p}{[}\PY{l+m+mf}{0.001}\PY{p}{,}\PY{l+m+mf}{0.1}\PY{p}{,}\PY{l+m+mi}{1}\PY{p}{,}\PY{l+m+mi}{10}\PY{p}{]}\PY{p}{\PYZcb{}}\PY{p}{,}
             \PY{l+s+s1}{\PYZsq{}}\PY{l+s+s1}{ET}\PY{l+s+s1}{\PYZsq{}}\PY{p}{:} \PY{p}{\PYZob{}} \PY{l+s+s1}{\PYZsq{}}\PY{l+s+s1}{n\PYZus{}estimators}\PY{l+s+s1}{\PYZsq{}}\PY{p}{:} \PY{p}{[}\PY{l+m+mi}{10}\PY{p}{,}\PY{l+m+mi}{100}\PY{p}{]}\PY{p}{,} \PY{l+s+s1}{\PYZsq{}}\PY{l+s+s1}{criterion}\PY{l+s+s1}{\PYZsq{}} \PY{p}{:} \PY{p}{[}\PY{l+s+s1}{\PYZsq{}}\PY{l+s+s1}{gini}\PY{l+s+s1}{\PYZsq{}}\PY{p}{,} \PY{l+s+s1}{\PYZsq{}}\PY{l+s+s1}{entropy}\PY{l+s+s1}{\PYZsq{}}\PY{p}{]} \PY{p}{,}\PY{l+s+s1}{\PYZsq{}}\PY{l+s+s1}{max\PYZus{}depth}\PY{l+s+s1}{\PYZsq{}}\PY{p}{:} \PY{p}{[}\PY{l+m+mi}{5}\PY{p}{,}\PY{l+m+mi}{50}\PY{p}{]}\PY{p}{,} \PY{l+s+s1}{\PYZsq{}}\PY{l+s+s1}{max\PYZus{}features}\PY{l+s+s1}{\PYZsq{}}\PY{p}{:} \PY{p}{[}\PY{l+s+s1}{\PYZsq{}}\PY{l+s+s1}{sqrt}\PY{l+s+s1}{\PYZsq{}}\PY{p}{,}\PY{l+s+s1}{\PYZsq{}}\PY{l+s+s1}{log2}\PY{l+s+s1}{\PYZsq{}}\PY{p}{]}\PY{p}{,}\PY{l+s+s1}{\PYZsq{}}\PY{l+s+s1}{min\PYZus{}samples\PYZus{}split}\PY{l+s+s1}{\PYZsq{}}\PY{p}{:} \PY{p}{[}\PY{l+m+mi}{2}\PY{p}{,}\PY{l+m+mi}{10}\PY{p}{]}\PY{p}{\PYZcb{}}\PY{p}{,}
             \PY{l+s+s1}{\PYZsq{}}\PY{l+s+s1}{AB}\PY{l+s+s1}{\PYZsq{}}\PY{p}{:} \PY{p}{\PYZob{}} \PY{l+s+s1}{\PYZsq{}}\PY{l+s+s1}{algorithm}\PY{l+s+s1}{\PYZsq{}}\PY{p}{:} \PY{p}{[}\PY{l+s+s1}{\PYZsq{}}\PY{l+s+s1}{SAMME}\PY{l+s+s1}{\PYZsq{}}\PY{p}{,} \PY{l+s+s1}{\PYZsq{}}\PY{l+s+s1}{SAMME.R}\PY{l+s+s1}{\PYZsq{}}\PY{p}{]}\PY{p}{,} \PY{l+s+s1}{\PYZsq{}}\PY{l+s+s1}{n\PYZus{}estimators}\PY{l+s+s1}{\PYZsq{}}\PY{p}{:} \PY{p}{[}\PY{l+m+mi}{1}\PY{p}{,}\PY{l+m+mi}{10}\PY{p}{,}\PY{l+m+mi}{100}\PY{p}{]}\PY{p}{\PYZcb{}}\PY{p}{,}
             \PY{l+s+s1}{\PYZsq{}}\PY{l+s+s1}{GB}\PY{l+s+s1}{\PYZsq{}}\PY{p}{:} \PY{p}{\PYZob{}}\PY{l+s+s1}{\PYZsq{}}\PY{l+s+s1}{n\PYZus{}estimators}\PY{l+s+s1}{\PYZsq{}}\PY{p}{:} \PY{p}{[}\PY{l+m+mi}{1}\PY{p}{]}\PY{p}{,} \PY{l+s+s1}{\PYZsq{}}\PY{l+s+s1}{learning\PYZus{}rate}\PY{l+s+s1}{\PYZsq{}} \PY{p}{:} \PY{p}{[}\PY{l+m+mf}{0.1}\PY{p}{]}\PY{p}{,}\PY{l+s+s1}{\PYZsq{}}\PY{l+s+s1}{subsample}\PY{l+s+s1}{\PYZsq{}} \PY{p}{:} \PY{p}{[}\PY{l+m+mf}{0.5}\PY{p}{]}\PY{p}{,} \PY{l+s+s1}{\PYZsq{}}\PY{l+s+s1}{max\PYZus{}depth}\PY{l+s+s1}{\PYZsq{}}\PY{p}{:} \PY{p}{[}\PY{l+m+mi}{5}\PY{p}{]}\PY{p}{\PYZcb{}}\PY{p}{,}
             \PY{l+s+s1}{\PYZsq{}}\PY{l+s+s1}{NB}\PY{l+s+s1}{\PYZsq{}} \PY{p}{:} \PY{p}{\PYZob{}}\PY{p}{\PYZcb{}}\PY{p}{,}
             \PY{l+s+s1}{\PYZsq{}}\PY{l+s+s1}{DT}\PY{l+s+s1}{\PYZsq{}}\PY{p}{:} \PY{p}{\PYZob{}}\PY{l+s+s1}{\PYZsq{}}\PY{l+s+s1}{criterion}\PY{l+s+s1}{\PYZsq{}}\PY{p}{:} \PY{p}{[}\PY{l+s+s1}{\PYZsq{}}\PY{l+s+s1}{gini}\PY{l+s+s1}{\PYZsq{}}\PY{p}{,} \PY{l+s+s1}{\PYZsq{}}\PY{l+s+s1}{entropy}\PY{l+s+s1}{\PYZsq{}}\PY{p}{]}\PY{p}{,} \PY{l+s+s1}{\PYZsq{}}\PY{l+s+s1}{max\PYZus{}depth}\PY{l+s+s1}{\PYZsq{}}\PY{p}{:} \PY{p}{[}\PY{l+m+mi}{1}\PY{p}{,}\PY{l+m+mi}{5}\PY{p}{,}\PY{l+m+mi}{10}\PY{p}{,}\PY{l+m+mi}{20}\PY{p}{]}\PY{p}{,} \PY{l+s+s1}{\PYZsq{}}\PY{l+s+s1}{max\PYZus{}features}\PY{l+s+s1}{\PYZsq{}}\PY{p}{:} \PY{p}{[}\PY{l+s+s1}{\PYZsq{}}\PY{l+s+s1}{sqrt}\PY{l+s+s1}{\PYZsq{}}\PY{p}{,}\PY{l+s+s1}{\PYZsq{}}\PY{l+s+s1}{log2}\PY{l+s+s1}{\PYZsq{}}\PY{p}{]}\PY{p}{,}\PY{l+s+s1}{\PYZsq{}}\PY{l+s+s1}{min\PYZus{}samples\PYZus{}split}\PY{l+s+s1}{\PYZsq{}}\PY{p}{:} \PY{p}{[}\PY{l+m+mi}{2}\PY{p}{,}\PY{l+m+mi}{5}\PY{p}{,}\PY{l+m+mi}{10}\PY{p}{]}\PY{p}{\PYZcb{}}\PY{p}{,}
             \PY{l+s+s1}{\PYZsq{}}\PY{l+s+s1}{LSVC}\PY{l+s+s1}{\PYZsq{}} \PY{p}{:}\PY{p}{\PYZob{}}\PY{l+s+s1}{\PYZsq{}}\PY{l+s+s1}{penalty}\PY{l+s+s1}{\PYZsq{}}\PY{p}{:} \PY{p}{[}\PY{l+s+s1}{\PYZsq{}}\PY{l+s+s1}{l1}\PY{l+s+s1}{\PYZsq{}}\PY{p}{,}\PY{l+s+s1}{\PYZsq{}}\PY{l+s+s1}{l2}\PY{l+s+s1}{\PYZsq{}}\PY{p}{]}\PY{p}{,} \PY{l+s+s1}{\PYZsq{}}\PY{l+s+s1}{C}\PY{l+s+s1}{\PYZsq{}} \PY{p}{:}\PY{p}{[}\PY{l+m+mf}{0.001}\PY{p}{,}\PY{l+m+mf}{0.01}\PY{p}{,}\PY{l+m+mf}{0.1}\PY{p}{,}\PY{l+m+mi}{1}\PY{p}{,}\PY{l+m+mi}{10}\PY{p}{]}\PY{p}{\PYZcb{}}\PY{p}{,}
             \PY{l+s+s1}{\PYZsq{}}\PY{l+s+s1}{KNN}\PY{l+s+s1}{\PYZsq{}} \PY{p}{:}\PY{p}{\PYZob{}}\PY{l+s+s1}{\PYZsq{}}\PY{l+s+s1}{n\PYZus{}neighbors}\PY{l+s+s1}{\PYZsq{}}\PY{p}{:} \PY{p}{[}\PY{l+m+mi}{1}\PY{p}{,}\PY{l+m+mi}{5}\PY{p}{,}\PY{l+m+mi}{10}\PY{p}{,}\PY{l+m+mi}{25}\PY{p}{]}\PY{p}{,}\PY{l+s+s1}{\PYZsq{}}\PY{l+s+s1}{weights}\PY{l+s+s1}{\PYZsq{}}\PY{p}{:} \PY{p}{[}\PY{l+s+s1}{\PYZsq{}}\PY{l+s+s1}{uniform}\PY{l+s+s1}{\PYZsq{}}\PY{p}{,}\PY{l+s+s1}{\PYZsq{}}\PY{l+s+s1}{distance}\PY{l+s+s1}{\PYZsq{}}\PY{p}{]}\PY{p}{,}\PY{l+s+s1}{\PYZsq{}}\PY{l+s+s1}{algorithm}\PY{l+s+s1}{\PYZsq{}}\PY{p}{:} \PY{p}{[}\PY{l+s+s1}{\PYZsq{}}\PY{l+s+s1}{auto}\PY{l+s+s1}{\PYZsq{}}\PY{p}{,}\PY{l+s+s1}{\PYZsq{}}\PY{l+s+s1}{ball\PYZus{}tree}\PY{l+s+s1}{\PYZsq{}}\PY{p}{,}\PY{l+s+s1}{\PYZsq{}}\PY{l+s+s1}{kd\PYZus{}tree}\PY{l+s+s1}{\PYZsq{}}\PY{p}{]}\PY{p}{\PYZcb{}}
                    \PY{p}{\PYZcb{}}
\end{Verbatim}


    \subsection{Part 4 and 5: Metrics, Plots and Comparison
Table}\label{part-4-and-5-metrics-plots-and-comparison-table}

    \subsubsection{I decided to call the do\_learning function with the
small and test grids. The function also produces precision-recall plots
for each model and each parameter
combination.}\label{i-decided-to-call-the-do_learning-function-with-the-small-and-test-grids.-the-function-also-produces-precision-recall-plots-for-each-model-and-each-parameter-combination.}

\subsubsection{Finally, the do\_learning function stores the results in
a CSV file which contains the model's especifications and metrics at
different threshold
levels.}\label{finally-the-do_learning-function-stores-the-results-in-a-csv-file-which-contains-the-models-especifications-and-metrics-at-different-threshold-levels.}

    \begin{Verbatim}[commandchars=\\\{\}]
{\color{incolor}In [{\color{incolor}25}]:} \PY{n}{results}\PY{p}{,} \PY{n}{matrix} \PY{o}{=} \PY{n}{do\PYZus{}learning}\PY{p}{(}\PY{n}{models\PYZus{}to\PYZus{}run}\PY{p}{,} \PY{n}{features}\PY{p}{,} \PY{l+s+s1}{\PYZsq{}}\PY{l+s+s1}{small}\PY{l+s+s1}{\PYZsq{}}\PY{p}{,} \PY{n}{X}\PY{p}{,} \PY{n}{y}\PY{p}{)}
\end{Verbatim}


    \begin{Verbatim}[commandchars=\\\{\}]
LR

    \end{Verbatim}

    
    \begin{verbatim}
<matplotlib.figure.Figure at 0x7f0f208eb7b8>
    \end{verbatim}

    
    \begin{center}
    \adjustimage{max size={0.9\linewidth}{0.9\paperheight}}{output_49_2.png}
    \end{center}
    { \hspace*{\fill} \\}
    
    
    \begin{verbatim}
<matplotlib.figure.Figure at 0x7f0ef7ff08d0>
    \end{verbatim}

    
    \begin{center}
    \adjustimage{max size={0.9\linewidth}{0.9\paperheight}}{output_49_4.png}
    \end{center}
    { \hspace*{\fill} \\}
    
    
    \begin{verbatim}
<matplotlib.figure.Figure at 0x7f0ef3490208>
    \end{verbatim}

    
    \begin{center}
    \adjustimage{max size={0.9\linewidth}{0.9\paperheight}}{output_49_6.png}
    \end{center}
    { \hspace*{\fill} \\}
    
    
    \begin{verbatim}
<matplotlib.figure.Figure at 0x7f0ef8bb0fd0>
    \end{verbatim}

    
    \begin{center}
    \adjustimage{max size={0.9\linewidth}{0.9\paperheight}}{output_49_8.png}
    \end{center}
    { \hspace*{\fill} \\}
    
    
    \begin{verbatim}
<matplotlib.figure.Figure at 0x7f0f1e9a7ef0>
    \end{verbatim}

    
    \begin{center}
    \adjustimage{max size={0.9\linewidth}{0.9\paperheight}}{output_49_10.png}
    \end{center}
    { \hspace*{\fill} \\}
    
    
    \begin{verbatim}
<matplotlib.figure.Figure at 0x7f0ef39780b8>
    \end{verbatim}

    
    \begin{center}
    \adjustimage{max size={0.9\linewidth}{0.9\paperheight}}{output_49_12.png}
    \end{center}
    { \hspace*{\fill} \\}
    
    
    \begin{verbatim}
<matplotlib.figure.Figure at 0x7f0ef3ddc4a8>
    \end{verbatim}

    
    \begin{center}
    \adjustimage{max size={0.9\linewidth}{0.9\paperheight}}{output_49_14.png}
    \end{center}
    { \hspace*{\fill} \\}
    
    
    \begin{verbatim}
<matplotlib.figure.Figure at 0x7f0ef3e57fd0>
    \end{verbatim}

    
    \begin{center}
    \adjustimage{max size={0.9\linewidth}{0.9\paperheight}}{output_49_16.png}
    \end{center}
    { \hspace*{\fill} \\}
    
    \begin{Verbatim}[commandchars=\\\{\}]
KNN

    \end{Verbatim}

    
    \begin{verbatim}
<matplotlib.figure.Figure at 0x7f0f1e9a7ef0>
    \end{verbatim}

    
    \begin{center}
    \adjustimage{max size={0.9\linewidth}{0.9\paperheight}}{output_49_19.png}
    \end{center}
    { \hspace*{\fill} \\}
    
    
    \begin{verbatim}
<matplotlib.figure.Figure at 0x7f0f26451198>
    \end{verbatim}

    
    \begin{center}
    \adjustimage{max size={0.9\linewidth}{0.9\paperheight}}{output_49_21.png}
    \end{center}
    { \hspace*{\fill} \\}
    
    
    \begin{verbatim}
<matplotlib.figure.Figure at 0x7f0f24b3c518>
    \end{verbatim}

    
    \begin{center}
    \adjustimage{max size={0.9\linewidth}{0.9\paperheight}}{output_49_23.png}
    \end{center}
    { \hspace*{\fill} \\}
    
    
    \begin{verbatim}
<matplotlib.figure.Figure at 0x7f0f063bbda0>
    \end{verbatim}

    
    \begin{center}
    \adjustimage{max size={0.9\linewidth}{0.9\paperheight}}{output_49_25.png}
    \end{center}
    { \hspace*{\fill} \\}
    
    
    \begin{verbatim}
<matplotlib.figure.Figure at 0x7f0f1e9a7ef0>
    \end{verbatim}

    
    \begin{center}
    \adjustimage{max size={0.9\linewidth}{0.9\paperheight}}{output_49_27.png}
    \end{center}
    { \hspace*{\fill} \\}
    
    
    \begin{verbatim}
<matplotlib.figure.Figure at 0x7f0f26451390>
    \end{verbatim}

    
    \begin{center}
    \adjustimage{max size={0.9\linewidth}{0.9\paperheight}}{output_49_29.png}
    \end{center}
    { \hspace*{\fill} \\}
    
    
    \begin{verbatim}
<matplotlib.figure.Figure at 0x7f0f1c0826a0>
    \end{verbatim}

    
    \begin{center}
    \adjustimage{max size={0.9\linewidth}{0.9\paperheight}}{output_49_31.png}
    \end{center}
    { \hspace*{\fill} \\}
    
    
    \begin{verbatim}
<matplotlib.figure.Figure at 0x7f0ef3e57fd0>
    \end{verbatim}

    
    \begin{center}
    \adjustimage{max size={0.9\linewidth}{0.9\paperheight}}{output_49_33.png}
    \end{center}
    { \hspace*{\fill} \\}
    
    
    \begin{verbatim}
<matplotlib.figure.Figure at 0x7f0ef7ff5668>
    \end{verbatim}

    
    \begin{center}
    \adjustimage{max size={0.9\linewidth}{0.9\paperheight}}{output_49_35.png}
    \end{center}
    { \hspace*{\fill} \\}
    
    
    \begin{verbatim}
<matplotlib.figure.Figure at 0x7f0f1c1687f0>
    \end{verbatim}

    
    \begin{center}
    \adjustimage{max size={0.9\linewidth}{0.9\paperheight}}{output_49_37.png}
    \end{center}
    { \hspace*{\fill} \\}
    
    
    \begin{verbatim}
<matplotlib.figure.Figure at 0x7f0ef8bb0fd0>
    \end{verbatim}

    
    \begin{center}
    \adjustimage{max size={0.9\linewidth}{0.9\paperheight}}{output_49_39.png}
    \end{center}
    { \hspace*{\fill} \\}
    
    
    \begin{verbatim}
<matplotlib.figure.Figure at 0x7f0f1e9a7ef0>
    \end{verbatim}

    
    \begin{center}
    \adjustimage{max size={0.9\linewidth}{0.9\paperheight}}{output_49_41.png}
    \end{center}
    { \hspace*{\fill} \\}
    
    
    \begin{verbatim}
<matplotlib.figure.Figure at 0x7f0f26451390>
    \end{verbatim}

    
    \begin{center}
    \adjustimage{max size={0.9\linewidth}{0.9\paperheight}}{output_49_43.png}
    \end{center}
    { \hspace*{\fill} \\}
    
    
    \begin{verbatim}
<matplotlib.figure.Figure at 0x7f0f26451550>
    \end{verbatim}

    
    \begin{center}
    \adjustimage{max size={0.9\linewidth}{0.9\paperheight}}{output_49_45.png}
    \end{center}
    { \hspace*{\fill} \\}
    
    
    \begin{verbatim}
<matplotlib.figure.Figure at 0x7f0ef8bb0fd0>
    \end{verbatim}

    
    \begin{center}
    \adjustimage{max size={0.9\linewidth}{0.9\paperheight}}{output_49_47.png}
    \end{center}
    { \hspace*{\fill} \\}
    
    
    \begin{verbatim}
<matplotlib.figure.Figure at 0x7f0f0635e6d8>
    \end{verbatim}

    
    \begin{center}
    \adjustimage{max size={0.9\linewidth}{0.9\paperheight}}{output_49_49.png}
    \end{center}
    { \hspace*{\fill} \\}
    
    
    \begin{verbatim}
<matplotlib.figure.Figure at 0x7f0f1e9e52e8>
    \end{verbatim}

    
    \begin{center}
    \adjustimage{max size={0.9\linewidth}{0.9\paperheight}}{output_49_51.png}
    \end{center}
    { \hspace*{\fill} \\}
    
    
    \begin{verbatim}
<matplotlib.figure.Figure at 0x7f0ef3e57fd0>
    \end{verbatim}

    
    \begin{center}
    \adjustimage{max size={0.9\linewidth}{0.9\paperheight}}{output_49_53.png}
    \end{center}
    { \hspace*{\fill} \\}
    
    
    \begin{verbatim}
<matplotlib.figure.Figure at 0x7f0f1c0826a0>
    \end{verbatim}

    
    \begin{center}
    \adjustimage{max size={0.9\linewidth}{0.9\paperheight}}{output_49_55.png}
    \end{center}
    { \hspace*{\fill} \\}
    
    
    \begin{verbatim}
<matplotlib.figure.Figure at 0x7f0f063bbda0>
    \end{verbatim}

    
    \begin{center}
    \adjustimage{max size={0.9\linewidth}{0.9\paperheight}}{output_49_57.png}
    \end{center}
    { \hspace*{\fill} \\}
    
    
    \begin{verbatim}
<matplotlib.figure.Figure at 0x7f0f1c0826a0>
    \end{verbatim}

    
    \begin{center}
    \adjustimage{max size={0.9\linewidth}{0.9\paperheight}}{output_49_59.png}
    \end{center}
    { \hspace*{\fill} \\}
    
    
    \begin{verbatim}
<matplotlib.figure.Figure at 0x7f0f063bbda0>
    \end{verbatim}

    
    \begin{center}
    \adjustimage{max size={0.9\linewidth}{0.9\paperheight}}{output_49_61.png}
    \end{center}
    { \hspace*{\fill} \\}
    
    
    \begin{verbatim}
<matplotlib.figure.Figure at 0x7f0ef3e57fd0>
    \end{verbatim}

    
    \begin{center}
    \adjustimage{max size={0.9\linewidth}{0.9\paperheight}}{output_49_63.png}
    \end{center}
    { \hspace*{\fill} \\}
    
    
    \begin{verbatim}
<matplotlib.figure.Figure at 0x7f0f063bbda0>
    \end{verbatim}

    
    \begin{center}
    \adjustimage{max size={0.9\linewidth}{0.9\paperheight}}{output_49_65.png}
    \end{center}
    { \hspace*{\fill} \\}
    
    \begin{Verbatim}[commandchars=\\\{\}]
DT

    \end{Verbatim}

    
    \begin{verbatim}
<matplotlib.figure.Figure at 0x7f0ef3e20470>
    \end{verbatim}

    
    \begin{center}
    \adjustimage{max size={0.9\linewidth}{0.9\paperheight}}{output_49_68.png}
    \end{center}
    { \hspace*{\fill} \\}
    
    
    \begin{verbatim}
<matplotlib.figure.Figure at 0x7f0f1e9a7ef0>
    \end{verbatim}

    
    \begin{center}
    \adjustimage{max size={0.9\linewidth}{0.9\paperheight}}{output_49_70.png}
    \end{center}
    { \hspace*{\fill} \\}
    
    
    \begin{verbatim}
<matplotlib.figure.Figure at 0x7f0f1c1687f0>
    \end{verbatim}

    
    \begin{center}
    \adjustimage{max size={0.9\linewidth}{0.9\paperheight}}{output_49_72.png}
    \end{center}
    { \hspace*{\fill} \\}
    
    
    \begin{verbatim}
<matplotlib.figure.Figure at 0x7f0ef7ff08d0>
    \end{verbatim}

    
    \begin{center}
    \adjustimage{max size={0.9\linewidth}{0.9\paperheight}}{output_49_74.png}
    \end{center}
    { \hspace*{\fill} \\}
    
    
    \begin{verbatim}
<matplotlib.figure.Figure at 0x7f0f0635e6d8>
    \end{verbatim}

    
    \begin{center}
    \adjustimage{max size={0.9\linewidth}{0.9\paperheight}}{output_49_76.png}
    \end{center}
    { \hspace*{\fill} \\}
    
    
    \begin{verbatim}
<matplotlib.figure.Figure at 0x7f0ef8bb0fd0>
    \end{verbatim}

    
    \begin{center}
    \adjustimage{max size={0.9\linewidth}{0.9\paperheight}}{output_49_78.png}
    \end{center}
    { \hspace*{\fill} \\}
    
    
    \begin{verbatim}
<matplotlib.figure.Figure at 0x7f0ef8bb0fd0>
    \end{verbatim}

    
    \begin{center}
    \adjustimage{max size={0.9\linewidth}{0.9\paperheight}}{output_49_80.png}
    \end{center}
    { \hspace*{\fill} \\}
    
    
    \begin{verbatim}
<matplotlib.figure.Figure at 0x7f0ef8bb0fd0>
    \end{verbatim}

    
    \begin{center}
    \adjustimage{max size={0.9\linewidth}{0.9\paperheight}}{output_49_82.png}
    \end{center}
    { \hspace*{\fill} \\}
    
    
    \begin{verbatim}
<matplotlib.figure.Figure at 0x7f0ef8bb0fd0>
    \end{verbatim}

    
    \begin{center}
    \adjustimage{max size={0.9\linewidth}{0.9\paperheight}}{output_49_84.png}
    \end{center}
    { \hspace*{\fill} \\}
    
    
    \begin{verbatim}
<matplotlib.figure.Figure at 0x7f0f1c168cc0>
    \end{verbatim}

    
    \begin{center}
    \adjustimage{max size={0.9\linewidth}{0.9\paperheight}}{output_49_86.png}
    \end{center}
    { \hspace*{\fill} \\}
    
    
    \begin{verbatim}
<matplotlib.figure.Figure at 0x7f0ef7ff5668>
    \end{verbatim}

    
    \begin{center}
    \adjustimage{max size={0.9\linewidth}{0.9\paperheight}}{output_49_88.png}
    \end{center}
    { \hspace*{\fill} \\}
    
    
    \begin{verbatim}
<matplotlib.figure.Figure at 0x7f0ef7ff5668>
    \end{verbatim}

    
    \begin{center}
    \adjustimage{max size={0.9\linewidth}{0.9\paperheight}}{output_49_90.png}
    \end{center}
    { \hspace*{\fill} \\}
    
    
    \begin{verbatim}
<matplotlib.figure.Figure at 0x7f0ef7ff5668>
    \end{verbatim}

    
    \begin{center}
    \adjustimage{max size={0.9\linewidth}{0.9\paperheight}}{output_49_92.png}
    \end{center}
    { \hspace*{\fill} \\}
    
    
    \begin{verbatim}
<matplotlib.figure.Figure at 0x7f0f1e9e5f28>
    \end{verbatim}

    
    \begin{center}
    \adjustimage{max size={0.9\linewidth}{0.9\paperheight}}{output_49_94.png}
    \end{center}
    { \hspace*{\fill} \\}
    
    
    \begin{verbatim}
<matplotlib.figure.Figure at 0x7f0f1e9e5f28>
    \end{verbatim}

    
    \begin{center}
    \adjustimage{max size={0.9\linewidth}{0.9\paperheight}}{output_49_96.png}
    \end{center}
    { \hspace*{\fill} \\}
    
    
    \begin{verbatim}
<matplotlib.figure.Figure at 0x7f0f1c168cc0>
    \end{verbatim}

    
    \begin{center}
    \adjustimage{max size={0.9\linewidth}{0.9\paperheight}}{output_49_98.png}
    \end{center}
    { \hspace*{\fill} \\}
    
    
    \begin{verbatim}
<matplotlib.figure.Figure at 0x7f0f1e9a7ef0>
    \end{verbatim}

    
    \begin{center}
    \adjustimage{max size={0.9\linewidth}{0.9\paperheight}}{output_49_100.png}
    \end{center}
    { \hspace*{\fill} \\}
    
    
    \begin{verbatim}
<matplotlib.figure.Figure at 0x7f0f1e9a7ef0>
    \end{verbatim}

    
    \begin{center}
    \adjustimage{max size={0.9\linewidth}{0.9\paperheight}}{output_49_102.png}
    \end{center}
    { \hspace*{\fill} \\}
    
    
    \begin{verbatim}
<matplotlib.figure.Figure at 0x7f0ef4b71160>
    \end{verbatim}

    
    \begin{center}
    \adjustimage{max size={0.9\linewidth}{0.9\paperheight}}{output_49_104.png}
    \end{center}
    { \hspace*{\fill} \\}
    
    
    \begin{verbatim}
<matplotlib.figure.Figure at 0x7f0f1e9e5f28>
    \end{verbatim}

    
    \begin{center}
    \adjustimage{max size={0.9\linewidth}{0.9\paperheight}}{output_49_106.png}
    \end{center}
    { \hspace*{\fill} \\}
    
    
    \begin{verbatim}
<matplotlib.figure.Figure at 0x7f0ef7ff5668>
    \end{verbatim}

    
    \begin{center}
    \adjustimage{max size={0.9\linewidth}{0.9\paperheight}}{output_49_108.png}
    \end{center}
    { \hspace*{\fill} \\}
    
    
    \begin{verbatim}
<matplotlib.figure.Figure at 0x7f0ef7ff5668>
    \end{verbatim}

    
    \begin{center}
    \adjustimage{max size={0.9\linewidth}{0.9\paperheight}}{output_49_110.png}
    \end{center}
    { \hspace*{\fill} \\}
    
    
    \begin{verbatim}
<matplotlib.figure.Figure at 0x7f0ef8bb0fd0>
    \end{verbatim}

    
    \begin{center}
    \adjustimage{max size={0.9\linewidth}{0.9\paperheight}}{output_49_112.png}
    \end{center}
    { \hspace*{\fill} \\}
    
    
    \begin{verbatim}
<matplotlib.figure.Figure at 0x7f0ef8bb0fd0>
    \end{verbatim}

    
    \begin{center}
    \adjustimage{max size={0.9\linewidth}{0.9\paperheight}}{output_49_114.png}
    \end{center}
    { \hspace*{\fill} \\}
    
    
    \begin{verbatim}
<matplotlib.figure.Figure at 0x7f0ef8bb0fd0>
    \end{verbatim}

    
    \begin{center}
    \adjustimage{max size={0.9\linewidth}{0.9\paperheight}}{output_49_116.png}
    \end{center}
    { \hspace*{\fill} \\}
    
    
    \begin{verbatim}
<matplotlib.figure.Figure at 0x7f0f1c1687f0>
    \end{verbatim}

    
    \begin{center}
    \adjustimage{max size={0.9\linewidth}{0.9\paperheight}}{output_49_118.png}
    \end{center}
    { \hspace*{\fill} \\}
    
    
    \begin{verbatim}
<matplotlib.figure.Figure at 0x7f0f1c1687f0>
    \end{verbatim}

    
    \begin{center}
    \adjustimage{max size={0.9\linewidth}{0.9\paperheight}}{output_49_120.png}
    \end{center}
    { \hspace*{\fill} \\}
    
    
    \begin{verbatim}
<matplotlib.figure.Figure at 0x7f0f1c1687f0>
    \end{verbatim}

    
    \begin{center}
    \adjustimage{max size={0.9\linewidth}{0.9\paperheight}}{output_49_122.png}
    \end{center}
    { \hspace*{\fill} \\}
    
    
    \begin{verbatim}
<matplotlib.figure.Figure at 0x7f0f1c1687f0>
    \end{verbatim}

    
    \begin{center}
    \adjustimage{max size={0.9\linewidth}{0.9\paperheight}}{output_49_124.png}
    \end{center}
    { \hspace*{\fill} \\}
    
    
    \begin{verbatim}
<matplotlib.figure.Figure at 0x7f0ef38cab38>
    \end{verbatim}

    
    \begin{center}
    \adjustimage{max size={0.9\linewidth}{0.9\paperheight}}{output_49_126.png}
    \end{center}
    { \hspace*{\fill} \\}
    
    
    \begin{verbatim}
<matplotlib.figure.Figure at 0x7f0f1c168cc0>
    \end{verbatim}

    
    \begin{center}
    \adjustimage{max size={0.9\linewidth}{0.9\paperheight}}{output_49_128.png}
    \end{center}
    { \hspace*{\fill} \\}
    
    
    \begin{verbatim}
<matplotlib.figure.Figure at 0x7f0ef8bb0fd0>
    \end{verbatim}

    
    \begin{center}
    \adjustimage{max size={0.9\linewidth}{0.9\paperheight}}{output_49_130.png}
    \end{center}
    { \hspace*{\fill} \\}
    
    
    \begin{verbatim}
<matplotlib.figure.Figure at 0x7f0f1c1687f0>
    \end{verbatim}

    
    \begin{center}
    \adjustimage{max size={0.9\linewidth}{0.9\paperheight}}{output_49_132.png}
    \end{center}
    { \hspace*{\fill} \\}
    
    
    \begin{verbatim}
<matplotlib.figure.Figure at 0x7f0f1e9e5f28>
    \end{verbatim}

    
    \begin{center}
    \adjustimage{max size={0.9\linewidth}{0.9\paperheight}}{output_49_134.png}
    \end{center}
    { \hspace*{\fill} \\}
    
    
    \begin{verbatim}
<matplotlib.figure.Figure at 0x7f0ef7ff08d0>
    \end{verbatim}

    
    \begin{center}
    \adjustimage{max size={0.9\linewidth}{0.9\paperheight}}{output_49_136.png}
    \end{center}
    { \hspace*{\fill} \\}
    
    
    \begin{verbatim}
<matplotlib.figure.Figure at 0x7f0f0635e6d8>
    \end{verbatim}

    
    \begin{center}
    \adjustimage{max size={0.9\linewidth}{0.9\paperheight}}{output_49_138.png}
    \end{center}
    { \hspace*{\fill} \\}
    
    
    \begin{verbatim}
<matplotlib.figure.Figure at 0x7f0ef8bb0fd0>
    \end{verbatim}

    
    \begin{center}
    \adjustimage{max size={0.9\linewidth}{0.9\paperheight}}{output_49_140.png}
    \end{center}
    { \hspace*{\fill} \\}
    
    
    \begin{verbatim}
<matplotlib.figure.Figure at 0x7f0ef4b71160>
    \end{verbatim}

    
    \begin{center}
    \adjustimage{max size={0.9\linewidth}{0.9\paperheight}}{output_49_142.png}
    \end{center}
    { \hspace*{\fill} \\}
    
    
    \begin{verbatim}
<matplotlib.figure.Figure at 0x7f0ef3e57fd0>
    \end{verbatim}

    
    \begin{center}
    \adjustimage{max size={0.9\linewidth}{0.9\paperheight}}{output_49_144.png}
    \end{center}
    { \hspace*{\fill} \\}
    
    
    \begin{verbatim}
<matplotlib.figure.Figure at 0x7f0f0635e6d8>
    \end{verbatim}

    
    \begin{center}
    \adjustimage{max size={0.9\linewidth}{0.9\paperheight}}{output_49_146.png}
    \end{center}
    { \hspace*{\fill} \\}
    
    
    \begin{verbatim}
<matplotlib.figure.Figure at 0x7f0f1e9e5f28>
    \end{verbatim}

    
    \begin{center}
    \adjustimage{max size={0.9\linewidth}{0.9\paperheight}}{output_49_148.png}
    \end{center}
    { \hspace*{\fill} \\}
    
    
    \begin{verbatim}
<matplotlib.figure.Figure at 0x7f0ef3e57fd0>
    \end{verbatim}

    
    \begin{center}
    \adjustimage{max size={0.9\linewidth}{0.9\paperheight}}{output_49_150.png}
    \end{center}
    { \hspace*{\fill} \\}
    
    
    \begin{verbatim}
<matplotlib.figure.Figure at 0x7f0ef3e57fd0>
    \end{verbatim}

    
    \begin{center}
    \adjustimage{max size={0.9\linewidth}{0.9\paperheight}}{output_49_152.png}
    \end{center}
    { \hspace*{\fill} \\}
    
    
    \begin{verbatim}
<matplotlib.figure.Figure at 0x7f0ef38cae10>
    \end{verbatim}

    
    \begin{center}
    \adjustimage{max size={0.9\linewidth}{0.9\paperheight}}{output_49_154.png}
    \end{center}
    { \hspace*{\fill} \\}
    
    
    \begin{verbatim}
<matplotlib.figure.Figure at 0x7f0f1e9e5f28>
    \end{verbatim}

    
    \begin{center}
    \adjustimage{max size={0.9\linewidth}{0.9\paperheight}}{output_49_156.png}
    \end{center}
    { \hspace*{\fill} \\}
    
    
    \begin{verbatim}
<matplotlib.figure.Figure at 0x7f0ef3e57fd0>
    \end{verbatim}

    
    \begin{center}
    \adjustimage{max size={0.9\linewidth}{0.9\paperheight}}{output_49_158.png}
    \end{center}
    { \hspace*{\fill} \\}
    
    
    \begin{verbatim}
<matplotlib.figure.Figure at 0x7f0ef3e57fd0>
    \end{verbatim}

    
    \begin{center}
    \adjustimage{max size={0.9\linewidth}{0.9\paperheight}}{output_49_160.png}
    \end{center}
    { \hspace*{\fill} \\}
    
    
    \begin{verbatim}
<matplotlib.figure.Figure at 0x7f0f1e9a7ef0>
    \end{verbatim}

    
    \begin{center}
    \adjustimage{max size={0.9\linewidth}{0.9\paperheight}}{output_49_162.png}
    \end{center}
    { \hspace*{\fill} \\}
    
    \begin{Verbatim}[commandchars=\\\{\}]
RF

    \end{Verbatim}

    
    \begin{verbatim}
<matplotlib.figure.Figure at 0x7f0ef3d91f60>
    \end{verbatim}

    
    \begin{center}
    \adjustimage{max size={0.9\linewidth}{0.9\paperheight}}{output_49_165.png}
    \end{center}
    { \hspace*{\fill} \\}
    
    
    \begin{verbatim}
<matplotlib.figure.Figure at 0x7f0ef38bff28>
    \end{verbatim}

    
    \begin{center}
    \adjustimage{max size={0.9\linewidth}{0.9\paperheight}}{output_49_167.png}
    \end{center}
    { \hspace*{\fill} \\}
    
    
    \begin{verbatim}
<matplotlib.figure.Figure at 0x7f0ef304cc50>
    \end{verbatim}

    
    \begin{center}
    \adjustimage{max size={0.9\linewidth}{0.9\paperheight}}{output_49_169.png}
    \end{center}
    { \hspace*{\fill} \\}
    
    
    \begin{verbatim}
<matplotlib.figure.Figure at 0x7f0f0635e6d8>
    \end{verbatim}

    
    \begin{center}
    \adjustimage{max size={0.9\linewidth}{0.9\paperheight}}{output_49_171.png}
    \end{center}
    { \hspace*{\fill} \\}
    
    
    \begin{verbatim}
<matplotlib.figure.Figure at 0x7f0ef375d2e8>
    \end{verbatim}

    
    \begin{center}
    \adjustimage{max size={0.9\linewidth}{0.9\paperheight}}{output_49_173.png}
    \end{center}
    { \hspace*{\fill} \\}
    
    
    \begin{verbatim}
<matplotlib.figure.Figure at 0x7f0f1c168cc0>
    \end{verbatim}

    
    \begin{center}
    \adjustimage{max size={0.9\linewidth}{0.9\paperheight}}{output_49_175.png}
    \end{center}
    { \hspace*{\fill} \\}
    
    
    \begin{verbatim}
<matplotlib.figure.Figure at 0x7f0f1c16d908>
    \end{verbatim}

    
    \begin{center}
    \adjustimage{max size={0.9\linewidth}{0.9\paperheight}}{output_49_177.png}
    \end{center}
    { \hspace*{\fill} \\}
    
    
    \begin{verbatim}
<matplotlib.figure.Figure at 0x7f0ef3d9b0b8>
    \end{verbatim}

    
    \begin{center}
    \adjustimage{max size={0.9\linewidth}{0.9\paperheight}}{output_49_179.png}
    \end{center}
    { \hspace*{\fill} \\}
    
    
    \begin{verbatim}
<matplotlib.figure.Figure at 0x7f0f1c16da20>
    \end{verbatim}

    
    \begin{center}
    \adjustimage{max size={0.9\linewidth}{0.9\paperheight}}{output_49_181.png}
    \end{center}
    { \hspace*{\fill} \\}
    
    
    \begin{verbatim}
<matplotlib.figure.Figure at 0x7f0ef3596780>
    \end{verbatim}

    
    \begin{center}
    \adjustimage{max size={0.9\linewidth}{0.9\paperheight}}{output_49_183.png}
    \end{center}
    { \hspace*{\fill} \\}
    
    
    \begin{verbatim}
<matplotlib.figure.Figure at 0x7f0ef3d91780>
    \end{verbatim}

    
    \begin{center}
    \adjustimage{max size={0.9\linewidth}{0.9\paperheight}}{output_49_185.png}
    \end{center}
    { \hspace*{\fill} \\}
    
    
    \begin{verbatim}
<matplotlib.figure.Figure at 0x7f0ef390a048>
    \end{verbatim}

    
    \begin{center}
    \adjustimage{max size={0.9\linewidth}{0.9\paperheight}}{output_49_187.png}
    \end{center}
    { \hspace*{\fill} \\}
    
    
    \begin{verbatim}
<matplotlib.figure.Figure at 0x7f0f26451198>
    \end{verbatim}

    
    \begin{center}
    \adjustimage{max size={0.9\linewidth}{0.9\paperheight}}{output_49_189.png}
    \end{center}
    { \hspace*{\fill} \\}
    
    
    \begin{verbatim}
<matplotlib.figure.Figure at 0x7f0ef7b615c0>
    \end{verbatim}

    
    \begin{center}
    \adjustimage{max size={0.9\linewidth}{0.9\paperheight}}{output_49_191.png}
    \end{center}
    { \hspace*{\fill} \\}
    
    
    \begin{verbatim}
<matplotlib.figure.Figure at 0x7f0ef7b615c0>
    \end{verbatim}

    
    \begin{center}
    \adjustimage{max size={0.9\linewidth}{0.9\paperheight}}{output_49_193.png}
    \end{center}
    { \hspace*{\fill} \\}
    
    
    \begin{verbatim}
<matplotlib.figure.Figure at 0x7f0ef7b61198>
    \end{verbatim}

    
    \begin{center}
    \adjustimage{max size={0.9\linewidth}{0.9\paperheight}}{output_49_195.png}
    \end{center}
    { \hspace*{\fill} \\}
    
    \begin{Verbatim}[commandchars=\\\{\}]
AB

    \end{Verbatim}

    
    \begin{verbatim}
<matplotlib.figure.Figure at 0x7f0ef7b615c0>
    \end{verbatim}

    
    \begin{center}
    \adjustimage{max size={0.9\linewidth}{0.9\paperheight}}{output_49_198.png}
    \end{center}
    { \hspace*{\fill} \\}
    
    
    \begin{verbatim}
<matplotlib.figure.Figure at 0x7f0ef390ab38>
    \end{verbatim}

    
    \begin{center}
    \adjustimage{max size={0.9\linewidth}{0.9\paperheight}}{output_49_200.png}
    \end{center}
    { \hspace*{\fill} \\}
    
    
    \begin{verbatim}
<matplotlib.figure.Figure at 0x7f0ef393cb00>
    \end{verbatim}

    
    \begin{center}
    \adjustimage{max size={0.9\linewidth}{0.9\paperheight}}{output_49_202.png}
    \end{center}
    { \hspace*{\fill} \\}
    
    
    \begin{verbatim}
<matplotlib.figure.Figure at 0x7f0f1c16d7b8>
    \end{verbatim}

    
    \begin{center}
    \adjustimage{max size={0.9\linewidth}{0.9\paperheight}}{output_49_204.png}
    \end{center}
    { \hspace*{\fill} \\}
    
    
    \begin{verbatim}
<matplotlib.figure.Figure at 0x7f0ef38fdba8>
    \end{verbatim}

    
    \begin{center}
    \adjustimage{max size={0.9\linewidth}{0.9\paperheight}}{output_49_206.png}
    \end{center}
    { \hspace*{\fill} \\}
    
    
    \begin{verbatim}
<matplotlib.figure.Figure at 0x7f0ef220af28>
    \end{verbatim}

    
    \begin{center}
    \adjustimage{max size={0.9\linewidth}{0.9\paperheight}}{output_49_208.png}
    \end{center}
    { \hspace*{\fill} \\}
    
    \begin{Verbatim}[commandchars=\\\{\}]
GB

    \end{Verbatim}

    
    \begin{verbatim}
<matplotlib.figure.Figure at 0x7f0ef38dc358>
    \end{verbatim}

    
    \begin{center}
    \adjustimage{max size={0.9\linewidth}{0.9\paperheight}}{output_49_211.png}
    \end{center}
    { \hspace*{\fill} \\}
    
    \begin{Verbatim}[commandchars=\\\{\}]
LSVC

    \end{Verbatim}

    
    \begin{verbatim}
<matplotlib.figure.Figure at 0x7f0ef3050e48>
    \end{verbatim}

    
    \begin{center}
    \adjustimage{max size={0.9\linewidth}{0.9\paperheight}}{output_49_214.png}
    \end{center}
    { \hspace*{\fill} \\}
    
    
    \begin{verbatim}
<matplotlib.figure.Figure at 0x7f0f063bbda0>
    \end{verbatim}

    
    \begin{center}
    \adjustimage{max size={0.9\linewidth}{0.9\paperheight}}{output_49_216.png}
    \end{center}
    { \hspace*{\fill} \\}
    
    
    \begin{verbatim}
<matplotlib.figure.Figure at 0x7f0f1e9a7ef0>
    \end{verbatim}

    
    \begin{center}
    \adjustimage{max size={0.9\linewidth}{0.9\paperheight}}{output_49_218.png}
    \end{center}
    { \hspace*{\fill} \\}
    
    
    \begin{verbatim}
<matplotlib.figure.Figure at 0x7f0ef8bb0fd0>
    \end{verbatim}

    
    \begin{center}
    \adjustimage{max size={0.9\linewidth}{0.9\paperheight}}{output_49_220.png}
    \end{center}
    { \hspace*{\fill} \\}
    
    
    \begin{verbatim}
<matplotlib.figure.Figure at 0x7f0f24b3c240>
    \end{verbatim}

    
    \begin{center}
    \adjustimage{max size={0.9\linewidth}{0.9\paperheight}}{output_49_222.png}
    \end{center}
    { \hspace*{\fill} \\}
    
    
    \begin{verbatim}
<matplotlib.figure.Figure at 0x7f0f1e9a7ef0>
    \end{verbatim}

    
    \begin{center}
    \adjustimage{max size={0.9\linewidth}{0.9\paperheight}}{output_49_224.png}
    \end{center}
    { \hspace*{\fill} \\}
    
    
    \begin{verbatim}
<matplotlib.figure.Figure at 0x7f0ef8bb0fd0>
    \end{verbatim}

    
    \begin{center}
    \adjustimage{max size={0.9\linewidth}{0.9\paperheight}}{output_49_226.png}
    \end{center}
    { \hspace*{\fill} \\}
    
    
    \begin{verbatim}
<matplotlib.figure.Figure at 0x7f0f063bbda0>
    \end{verbatim}

    
    \begin{center}
    \adjustimage{max size={0.9\linewidth}{0.9\paperheight}}{output_49_228.png}
    \end{center}
    { \hspace*{\fill} \\}
    
    
    \begin{verbatim}
<matplotlib.figure.Figure at 0x7f0f1e9a7ef0>
    \end{verbatim}

    
    \begin{center}
    \adjustimage{max size={0.9\linewidth}{0.9\paperheight}}{output_49_230.png}
    \end{center}
    { \hspace*{\fill} \\}
    
    
    \begin{verbatim}
<matplotlib.figure.Figure at 0x7f0ef8bb0fd0>
    \end{verbatim}

    
    \begin{center}
    \adjustimage{max size={0.9\linewidth}{0.9\paperheight}}{output_49_232.png}
    \end{center}
    { \hspace*{\fill} \\}
    
    \subsection{Results Data Frame}\label{results-data-frame}

    \begin{Verbatim}[commandchars=\\\{\}]
{\color{incolor}In [{\color{incolor}27}]:} \PY{n}{results}\PY{o}{.}\PY{n}{head}\PY{p}{(}\PY{p}{)}
\end{Verbatim}


\begin{Verbatim}[commandchars=\\\{\}]
{\color{outcolor}Out[{\color{outcolor}27}]:}   model\_type                                                clf  \textbackslash{}
         0         LR  LogisticRegression(C=10, class\_weight=None, du{\ldots}   
         1         LR  LogisticRegression(C=10, class\_weight=None, du{\ldots}   
         2         LR  LogisticRegression(C=10, class\_weight=None, du{\ldots}   
         3         LR  LogisticRegression(C=10, class\_weight=None, du{\ldots}   
         4         LR  LogisticRegression(C=10, class\_weight=None, du{\ldots}   
         
                               parameters  train\_time  predict\_time   auc-roc  \textbackslash{}
         0  \{'penalty': 'l1', 'C': 0.001\}    2.050980      2.630951  0.674300   
         1  \{'penalty': 'l2', 'C': 0.001\}    1.912815      1.861484  0.649025   
         2    \{'penalty': 'l1', 'C': 0.1\}   18.176160     33.404562  0.694174   
         3    \{'penalty': 'l2', 'C': 0.1\}    1.834864      1.855404  0.649025   
         4      \{'penalty': 'l1', 'C': 1\}   11.038425     17.124494  0.694306   
         
              p\_at\_5   p\_at\_10   p\_at\_20    r\_at\_5   r\_at\_10   r\_at\_20   f1\_at\_5  \textbackslash{}
         0  0.180267  0.150133  0.128000  0.132705  0.221044  0.376914  0.152872   
         1  0.134933  0.124000  0.113333  0.099333  0.182568  0.333726  0.114428   
         2  0.334933  0.240267  0.162667  0.246565  0.353750  0.478995  0.284034   
         3  0.134933  0.124000  0.113333  0.099333  0.182568  0.333726  0.114428   
         4  0.334400  0.240000  0.162533  0.246172  0.353357  0.478602  0.283582   
         
            f1\_at\_10  f1\_at\_20  
         0  0.178815  0.191102  
         1  0.147689  0.169205  
         2  0.286168  0.242859  
         3  0.147689  0.169205  
         4  0.285850  0.242660  
\end{Verbatim}
            
    \subsection{Summary}\label{summary}

    \begin{itemize}
\tightlist
\item
  Logistic Regression: Better with l1 penalty and C = 10
\item
  Decision Tree works better with a depth equal to 5
\item
  RandomForests with sqrt method appear to be better
\end{itemize}

    \subsubsection{It is clear that the test grid produces terrible results
in terms of predicition, but it is useful to show the
work.}\label{it-is-clear-that-the-test-grid-produces-terrible-results-in-terms-of-predicition-but-it-is-useful-to-show-the-work.}

    \begin{Verbatim}[commandchars=\\\{\}]
{\color{incolor}In [{\color{incolor}29}]:} \PY{n}{results\PYZus{}test}\PY{p}{,} \PY{n}{matrix\PYZus{}test} \PY{o}{=} \PY{n}{do\PYZus{}learning}\PY{p}{(}\PY{n}{models\PYZus{}to\PYZus{}run}\PY{p}{,} \PY{n}{features}\PY{p}{,} \PY{l+s+s1}{\PYZsq{}}\PY{l+s+s1}{test}\PY{l+s+s1}{\PYZsq{}}\PY{p}{,} \PY{n}{X}\PY{p}{,} \PY{n}{y}\PY{p}{)}
\end{Verbatim}


    \begin{Verbatim}[commandchars=\\\{\}]
LR

    \end{Verbatim}

    
    \begin{verbatim}
<matplotlib.figure.Figure at 0x7f0ee0a49eb8>
    \end{verbatim}

    
    \begin{center}
    \adjustimage{max size={0.9\linewidth}{0.9\paperheight}}{output_55_2.png}
    \end{center}
    { \hspace*{\fill} \\}
    
    \begin{Verbatim}[commandchars=\\\{\}]
KNN

    \end{Verbatim}

    
    \begin{verbatim}
<matplotlib.figure.Figure at 0x7f0ef375d0f0>
    \end{verbatim}

    
    \begin{center}
    \adjustimage{max size={0.9\linewidth}{0.9\paperheight}}{output_55_5.png}
    \end{center}
    { \hspace*{\fill} \\}
    
    \begin{Verbatim}[commandchars=\\\{\}]
DT

    \end{Verbatim}

    
    \begin{verbatim}
<matplotlib.figure.Figure at 0x7f0ee09660f0>
    \end{verbatim}

    
    \begin{center}
    \adjustimage{max size={0.9\linewidth}{0.9\paperheight}}{output_55_8.png}
    \end{center}
    { \hspace*{\fill} \\}
    
    \begin{Verbatim}[commandchars=\\\{\}]
RF

    \end{Verbatim}

    
    \begin{verbatim}
<matplotlib.figure.Figure at 0x7f0ef37519e8>
    \end{verbatim}

    
    \begin{center}
    \adjustimage{max size={0.9\linewidth}{0.9\paperheight}}{output_55_11.png}
    \end{center}
    { \hspace*{\fill} \\}
    
    \begin{Verbatim}[commandchars=\\\{\}]
AB

    \end{Verbatim}

    
    \begin{verbatim}
<matplotlib.figure.Figure at 0x7f0ee1482400>
    \end{verbatim}

    
    \begin{center}
    \adjustimage{max size={0.9\linewidth}{0.9\paperheight}}{output_55_14.png}
    \end{center}
    { \hspace*{\fill} \\}
    
    \begin{Verbatim}[commandchars=\\\{\}]
GB

    \end{Verbatim}

    
    \begin{verbatim}
<matplotlib.figure.Figure at 0x7f0ef3978f60>
    \end{verbatim}

    
    \begin{center}
    \adjustimage{max size={0.9\linewidth}{0.9\paperheight}}{output_55_17.png}
    \end{center}
    { \hspace*{\fill} \\}
    
    \begin{Verbatim}[commandchars=\\\{\}]
LSVC

    \end{Verbatim}

    
    \begin{verbatim}
<matplotlib.figure.Figure at 0x7f0ecfdde278>
    \end{verbatim}

    
    \begin{center}
    \adjustimage{max size={0.9\linewidth}{0.9\paperheight}}{output_55_20.png}
    \end{center}
    { \hspace*{\fill} \\}
    
    \subsection{Confusion Party}\label{confusion-party}

    \subsubsection{As an additional tool for model comparison, I created the
following subplots for the confusion matrix for each
model}\label{as-an-additional-tool-for-model-comparison-i-created-the-following-subplots-for-the-confusion-matrix-for-each-model}

    \begin{Verbatim}[commandchars=\\\{\}]
{\color{incolor}In [{\color{incolor}30}]:} \PY{n}{confusion\PYZus{}party}\PY{p}{(}\PY{n}{matrix\PYZus{}test}\PY{p}{,} \PY{p}{[}\PY{l+s+s1}{\PYZsq{}}\PY{l+s+s1}{Non\PYZhy{}Deliquents}\PY{l+s+s1}{\PYZsq{}}\PY{p}{,} \PY{l+s+s1}{\PYZsq{}}\PY{l+s+s1}{Deliquents}\PY{l+s+s1}{\PYZsq{}}\PY{p}{]}\PY{p}{)}
\end{Verbatim}


    \begin{center}
    \adjustimage{max size={0.9\linewidth}{0.9\paperheight}}{output_58_0.png}
    \end{center}
    { \hspace*{\fill} \\}
    
    \begin{Verbatim}[commandchars=\\\{\}]
{\color{incolor}In [{\color{incolor}31}]:} \PY{n}{results}\PY{o}{.}\PY{n}{to\PYZus{}csv}\PY{p}{(}\PY{l+s+s1}{\PYZsq{}}\PY{l+s+s1}{results.csv}\PY{l+s+s1}{\PYZsq{}}\PY{p}{,} \PY{n}{index}\PY{o}{=}\PY{k+kc}{False}\PY{p}{)}
\end{Verbatim}



    % Add a bibliography block to the postdoc
    
    
    
    \end{document}
